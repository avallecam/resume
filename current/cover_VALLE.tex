% Cover letter using letter.sty
\documentclass{letter} 					% Uses 10pt
% Use \documentstyle[newcent]{letter} for New Century Schoolbook postscript font

\usepackage{graphicx}
\usepackage{hyperref}
\hypersetup{
	colorlinks=true,
	linkcolor=black,
	filecolor=balck,      
	urlcolor=black,
}
%%%%%%%%%%%%%%%%%%%%%%%%%%%%%%%%%%%%%%%%%%%%
%% ADD-IN to activate spanish text (tildes) [UTIL PARA FECHAS]
%\usepackage[spanish]{babel}
%\selectlanguage{spanish}
%\usepackage[utf8]{inputenc}
%%%%%%%%%%%%%%%%%%%%%%%%%%%%%%%%%%%%%%%%%%%%


% the following commands control the margins:
\topmargin=-1in    						% Make letterhead start about 1 inch from top of page 
\textheight=8.5in  						% text height can be bigger for a longer letter
\oddsidemargin=0pt 						% leftmargin is 1 inch
\textwidth=6.5in   						% textwidth of 6.5in leaves 1 inch for right margin

\begin{document}

\signature{Andree Valle Campos, MSc(c).}         % name for signature 
\longindentation=0pt                    % needed to get closing flush left
\let\raggedleft\raggedright             % needed to get date flush left
 
 
\begin{letter}{
		
		Malaria: Parasites \& Hosts Unit\\
%		Facultad de Salud P\'ublica y Administraci\'on\\
		Intitut Pasteur\\
		Paris, France
%		Ms. Terri Roberts \\
%		Senior Staff Recruiter \\
%		XYZ Corporation \\
%		Rt. 56 \\
%		Anytown, New Jersey 05867
		}


\begin{flushleft}
{\large\bf Andree Valle Campos, MSc(c).}
\end{flushleft}
%\medskip
\hrule height 1pt
\begin{flushright}
%\hfill @avallecam, avallecam@gmail.com, (+51)~950~951~722 \\
\hfill \href{https://gist.github.com/avallecam}{\includegraphics[scale=.005]{../figure/gh_logo.png}} \href{https://stackoverflow.com/users/6702544/avallecam}{\includegraphics[scale=.035]{../figure/so_logo.png}} \href{https://twitter.com/avallecam}{\includegraphics[scale=.015]{../figure/tw_logo.jpg} @avallecam} ~ \includegraphics[scale=.02]{../figure/gm_logo.png} avallecam@gmail.com ~  \includegraphics[scale=.015]{../figure/wa_logo.jpg} (+51)~950~951~722\\
\hfill Calle Tambo Hu\'ascar 201, San Miguel. Lima - Per\'u
\end{flushright} 
\vfill % forces letterhead to top of page

% short bio:
% Andree Valle-Campos
% Bachiller en Genética y Biotecnología egresado de la Universidad 
% Nacional Mayor de San Marcos (UNMSM) con tres años de experiencia 
% en la investigación genómica y bioestadística aplicada a la 
% Inmunología y Epidemiología. He sido practicante y tesista en el 
% Departamento de Parasitología del Centro de Investigación de 
% Enfermedades Tropicales de la Marina de los Estados Unidos (NAMRU-6). 
% También he colaboradorado en el Grupo de Estructuras Naturales e 
% Investigación Teórica y en el Laboratorio de Fisiología de la 
% Reproducción Animal, ambos de la Facultad de Ciencias Biológicas 
% de la UNMSM. Actualmente soy consultor de proyectos colaborativos 
% entre el NAMRU-6, la Universidad Nacional de la Amazonía Peruana y 
% la Universidad Peruana Cayetano Heredia.
%

% Andree Valle Campos, UNMSM-PERÚ.
% 
% Bachiller en Genética y Biotecnología egresado de la Universidad 
% Nacional Mayor de San Marcos (UNMSM). Actualmente se desempeña como 
% consultor en proyectos colaborativos entre la Universidad Nacional 
% de la Amazonía Peruana (UNAP), la Universidad Peruana Cayetano Heredia 
% (UPCH) y el Centro de Investigación de Enfermedades Tropicales de 
% la Marina de los Estados Unidos (NAMRU-6).
% 
% Cuenta con experiencia en la investigación bioinformática, serológica 
% y epidemiológica aplicada al estudio de la Malaria, principalmente en 
% el perfil a escala proteómica de la respuesta de anticuerpos ante la 
% malaria vivax. Posee conocimientos en el análisis de microarreglos, 
% ciencia de datos, ciencia reproducible y herramientas de software libre.
% 
% Ha sido practicante y tesista en el Departamento de Parasitología 
% del NAMRU-6, donde implementó pipelines para la estandarización, modelamiento 
% y visualización de inmunoensayos tradicionales y high-throughput screenings. 
% También ha colaboradorado en el Grupo de Estructuras Naturales e 
% Investigación Teórica y en el Laboratorio de Fisiología de la Reproducción 
% Animal, ambos de la Facultad de Ciencias Biológicas de la UNMSM.

%aHEADLINE
%Quantitative-oriented Biologist skilled in high-dimensional data analysis
%
%aPROFILE
%
%aEXPERIENCE
%
%At NAMRU-6, I implemented data analysis tools for immunoassays, of large- and short-scale, 
%relevant for Malaria surveillance and research. 
%Using R as statistical computing software, I developed reproducible workflows for the 
%(i) standardization of high-throughput ELISA assays for sero-surveillance, 
%(ii) modeling and visualization of Protein microarrays for large-scale Antibody Profiles 
%in response to infections with applications on vaccine candidates discovery, and 
%(iii) short-scale Dose-reponse assays to test drug resistance or total parasite biomass.
%
%At my home university, I gained bench experience doing laboratory research in a hot topic: 
%Magnetic field effects on animal reproduction. 
%Working hand-in-hand with Electrical Engineers, a novel and custom experimental setting 
%was developed, with induction coils and a humidity chamber, 
%to assess in vitro how sperm quality response to a extremely low frequency field. 
%From the writing proposal to a conference poster presentation, almost all the steps of 
%research activity were covered.
%
%aSUMMARY
%
%Andree Valle-Campos is a quantitative-oriented biologist highly interested in interdisciplinary research. 
%Working in the interphase between Epidemiology and large-scale Immunology, 
%he became fluent in R and enthusiastic for more computational applications 
%into its daily research. In parallel, he is also keen on applying and teaching formal approaches 
%aimed to understand the inner logic of biological systems.


%aCOVER LETTER
%
%***

 
\opening{To whom it may concern :} 
%\opening{A quien corresponda:} 
 
%\noindent 
%My name is Andree Valle Campos and 
%I am a BSc in Genetics and Biotechnology with 3 years of experience on computational research. 
%after receiving this degree. %of independent
%\noindent 
%Le escribo con el objetivo de expresar mi interés en la aplicación a la Beca EMERGE para la Maestría en Ciencias en Investigación Epidemiológica que ustedes ofrecen, la cual tuve conocimiento por redes sociales.

\noindent
I write to you in order to express my interest in the application for the internship that The Malaria: Parasites \& Hosts Unit at Institut Pasteur is offering, which I learned from your twitter account (\href{https://twitter.com/michaelwhite\_36/status/1177487514174537728?s=03}{link}).

\noindent 
I introduce myself as a researcher skilled in data science with a 4 year experience in biostatistics, bioinformatic and epidemiological research on Malaria.
As a masters candidate in epidemiological research, my thesis is on asymptomatic malaria and human movement behavior. Even though, as you can gather from my CV, my previous research was focused on the discovery of malaria immunogenic antigens through protein microarrays and the estimation of disease transmission through serological surveys.

\noindent 
It is due to this last research topic that I had reference on your work. In that research project I was in charge of mostly all the data analysis workflow: standardization of ELISA plates, estimation of arbitrary antibody titers using log-logistic regression models, classification of distributions by gaussian mixture models and estimation of seroconvertion rates with reversible serocatalitic models. All of this experience ended defining myself as an applied quantitative researcher and a publication at the Malaria Journal.

\noindent 
With this internship opportunity, I would like to re-open this research branch and get deep in serological surveys, adding a stronger modelling perspective on the evaluation of health interventions. Even more, this would be a strong step in my career that would improve my possibilities to be accepted in competitive international PhD programs. In there I would like to explore the usage of new technologies as remote lab-on-a-chip sensors to make surveillance cheaper in human workforce, more informative on molecular targets, and faster in data acquisition to tell decision makers.

\noindent
I would be very grateful to request an interview and talk in more detail on the issue. My schedule is flexible enough to set a meeting at any hour according to your availability. Thanks for your time to consider my credentials.

\closing{Sincerely yours,} 

\encl{Curriculum Vitae (cv-20190930-valle\_campos\_andree.pdf)}  				% Enclosures


%As you can gather from my CV, during the last two years I self-learned and trained in R and two of its major environments: \textit{Bioconductor} and the \textit{Tidyverse}, in order to implement workflows for genome-scale experiments and integrate them with raw epidemiological datasets. Furthermore, \textit{Rmarkdown notebooks} allowed me to generate scientific reports that improved the quality and traceability of my procedures.

%Me presento como un biólogo hábil en el análisis de datos con tres años de experiencia en la investigación bioinformática.
%I introduce myself as a biologist skilled in Data Science with 3 years of experience on bioinformatic research.
%I introduce myself as a quantitative-oriented biologist with 3 years of experience on computational research. 
%My interest in the position came from both: the host-pathogen interactomic approach through the potato virome, and the possibility to optimize this procedure by the implementation of a surveillance-like automated pipeline to process, analyze and visualize molecular data using free software.
%My experience as an omics-data pipeline developer came from my work in the interface between epidemiology and immunology.

%As you can read from my CV, I have a strong experience in the fields of Genomics Data Science and Bioinformatics. 
%While working in the interphase between Epidemiology and large-scale Immunology, I implemented reproducible data analysis workflows and, as a result, gained fluency in R, several programming languages and other computational tools that optimized my daily research.Furthermore, formal learning in fields like phylogenetics and structural genomics, gave mea in-depth understanding and critical perspective in the required field.
%I was immersed in the problematic around immunoassay non-reproducible data analysis methods. 
%In order to solve this issue, I developed two main reproducible workflows for current short-scale and future routine high-throughput assays. 
%These improvements would have a relevant impact in malaria sero-surveillance and vaccine development protocols.
%\noindent 
%This experience introduced me to the problem of 'big data' in immunology. 
%For this reason, a next step in my career is to contribute with multiple source data integration methods. 
%This approach would allow us not only to read, but to understand biological data through modeling. 
%That is why I am very interested in your awesome efforts on automated biological experiments using microfluidics. 
%Although minimalistic, this improvement would allow us to measure different sources of information, 
%from intracellular to extracellular responses, and estimate their variability in a more controlled fashion. 
%I thought that the CZ BioHub may have the great environment to further study this hypothesis.

%\noindent
%With respect to the specific requirements, I have also received formal training in the traditional toolbox to analyze Genomics data from both my undergraduate studies and complementary international courses. For instance, at the Wellcome Trust advanced course I improved my knowledge on the application and interpretation of terminal-based and Galaxy server workflows for RNA-seq data.
%Secondly, although computationally biased, I have also accomplished molecular laboratory competences through hands-on courses with an outstanding participation in the summer of 2013, gaining an invitation to a complementary workshop in November as a prize.%analytic techniques?
%Finally, even though I am not in a Master Degree Program yet, I am willing to move on and start one the next year, looking with interest either lab-based programs as the Biochemistry and Molecular Biology Master given by the Universidad Peruana Cayetano Heredia (UPCH) or computationally-based as the Data Science Master of the Ricardo Palma University (URP). Starting the project in advance would give me the chance to focus extra-time on key research aspects like literature review, project design and methods, including the Statistical Analysis Plan.

%\noindent 
%As you can also gather from my CV, I have participated in schools and courses in different countries, all involving interdisciplinary work and hands-on training with internationally recognized researchers. 
%From these experiences I expanded my scientific horizons, improved my communication abilities and openness to other cultures. 
%I also gave University classes for three years, sharing some of my learnings and encouraging new people into theoretical and computational biology.

%\noindent 
%As an active self-learner and enthusiast for group-oriented research, I am extremely sure that a challenging environment as the CIP would improve my gained background.
%I believe that my presence in your institution would be a great benefit to everyone. 
%Even more, this would be a strong step in my career that would improve my possibilities to be accepted in competitive international PhD programs.
%I am enthusiastic about the possibility of working together and I hope to hear from you very soon.
 
%***
%
%aGRAND LOCUS Notes:
%
%(1) Say immediately what stage of your career you are. 
%This is particularly important if you apply to an open call, as you will compete with hundreds of applicants and readers will have very little time to read your letter.
%
%(2) Tell your domain(s) of expertise. 
%By the third sentence, the reader must know how much experience you have (this is the previous note) and what you are good at.
%
%(3) Explain your choices. 
%Do not hesitate to say what you like, what fascinates you. 
%The reader will most likely be a scientist too, so genuine enthusiasm for your research will work better than career objectives.
%
%(4) Say what you want to do next. 
%This shows that you know where you are going.
%
%(5) Show some interest for the lab. 
%Check their website and their research lines. Mention this in your text along with some published work. 
%Propose something, even if this is naive or preliminary.
%
%(6) Interpret your CV. 
%This may seem obvious, but you can use the end of the cover letter to explain how your experience demonstrates your other qualities. 


%aMORE TEMPLATE NOTES
%
%aPARAGRAPH ONE: State reason for letter, name the position or type 
%of work you are applying for and identify source from  which  you 
%learned   of   the  opening.  (i.e.  Career  Development  Center, 
%newspaper, employment service, personal contact). 
% 
%aPARAGRAPH  TWO:  Indicate why you are interested in the position, 
%the company, its products, services - above all, stress what  you 
%can  do  for  the employer. If you are a recent graduate, explain 
%how your academic background makes you a qualified candidate  for 
%the  position.  If  you have practical work experience, point out 
%specific achievements or unique qualifications. Try not to repeat 
%the  same  information  the reader will find in the resume. Refer 
%the reader to the enclosed resume or application which summarizes 
%your  qualifications,  training,  and experiences. The purpose of 
%this section is to strengthen your resume  by  providing  details 
%which bring your experiences to life. 
% 
%aPARAGRAPH THREE: Request a personal interview and  indicate  your 
%flexibility as to the time and place. Repeat your phone number in 
%the letter and offer assistance to help in a speedy response. For 
%example,  state that you will be in the city where the company is 
%located on a certain date and would like to set up an  interview. 
%Or,  state  that  you  will  call  on a certain date to set up an 
%interview. End the letter by thanking  the  employer  for  taking 
%time to consider your credentials. 

% a SEND example for DATA WRANGLER JOB

% Dr. Hans Ekbrand,
% 
% I write to you in order to express my interest in the application 
% for the R data wrangler consultant as a freelance job, which I learned 
% from the web page of Jobs for R-users (https://www.r-users.com/jobs/r-data-wrangler/).
% 
% I introduce myself as a biologist skilled in Data Science with 3 years 
% of experience on bioinformatic research. My interest in the position 
% came from both: the use open public datasets of demographic and health 
% data, and the application of reproducibility principles for its analysis 
% with free software. My experience as data wrangler came from my work in 
% the interfase between epidemiology and immunology. As you can gather 
% from my CV, during the last two years I self-learned and acquire fluency 
% in R and two of its major environments: Bioconductor and the Tidyverse, 
% in order to implement workflows for genome-scale experiments and integrate 
% them with raw epidemiological datasets that requiered the amazing tools 
% provided by the last one. Furthermore, Rmarkdown notebooks allowed me to 
% be polyglot by using bash (terminal) commands inside them when required.
% 
% With respect to the specific requierements, I can propose tidy alternatives 
% to the import SPSS files using haven::read_spss() 
% [https://www.rdocumentation.org/packages/haven/versions/1.0.0], and the 
% usage of car::recode() with forcats::fct_recode() 
% [https://www.rdocumentation.org/packages/forcats/versions/0.2.0]. I am 
% confident that this work would allow me to improve my problem-solving 
% skills, code readability and give me the grate opportunity to start my 
% first coding collaborative work. The gained experience would also show me 
% how to apply this approach in my current research field and achieve the aim 
% to reduce the gap between data acquisition and responsiveness of the public 
% health system, in the same way as recent experiences in other fields have 
% shown (https://www.nature.com/articles/s41559-017-0160).
% 
% I would be very grateful to request an interview and talk in more detail 
% on the issue. My schedule is flexible enough to set a meeting at a hour 
% according to your timezone (I am in GMT+5). Thanks for your time to consider 
% my credentials.
% 
% Sincerely yours,
% 
% Andree Valle Campos
% 
% attached: Curriculum Vitae (CV_ValleCampos-Andree.pdf)

\end{letter}
 

\end{document}






