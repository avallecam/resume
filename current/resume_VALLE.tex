\documentclass[margin,line]{res}
%\usepackage{fontawesome} %https://tex.stackexchange.com/questions/190927/linkedin-logo-in-latex
\usepackage{graphicx}
\renewcommand{\namefont}{\bfseries\LARGE}

\oddsidemargin -.5in
\evensidemargin -.5in
\textwidth=6.0in
\itemsep=0in
\parsep=0in
\topmargin=0in
\topskip=0in
\textheight 10in
 
\newenvironment{list1}{
  \begin{list}{\ding{113}}{%
      \setlength{\itemsep}{0in}
      \setlength{\parsep}{0in} \setlength{\parskip}{0in}
      \setlength{\topsep}{0in} \setlength{\partopsep}{0in}
      \setlength{\leftmargin}{0.17in}}}{\end{list}}
\newenvironment{list2}{
  \begin{list}{$\bullet$}{%
      \setlength{\itemsep}{0in}
      \setlength{\parsep}{0in} \setlength{\parskip}{0in}
      \setlength{\topsep}{0in} \setlength{\partopsep}{0in}
      \setlength{\leftmargin}{0.2in}}}{\end{list}}
    
\begin{document}
%\newlength{\maxmarg}
\name{\LARGE Andree Valle Campos}
\address{\includegraphics[scale=.62]{../figure/g_logo.png} \includegraphics[scale=.38]{../figure/s_logo.png} \includegraphics[scale=.16]{../figure/t_logo.png} @avallecam ~ \includegraphics[scale=.6]{../figure/m_logo.png} avallecam@gmail.com ~  \includegraphics[scale=.45]{../figure/w_logo.jpg} (+51)~950~951~722}
\address{%\hspace{20\maxmarg}
	~~~~~~~~~~~~~~~~~~~~~~~Calle Tambo Huascar 201, San Miguel. Lima - Peru}
%\address{\textit{twitter:} @avallecam}
%\includegraphics[scale=.42]{../figure/g_logo_2.png}

\begin{resume}

\vspace*{.15in}

\section{\sc Research Interests}% %

Data science approaches to analysis and visualization in Genomics, Inmunology and Epidemiology.\\
Quantitative biology, Reproducible science, Free software solutions, and Open science projects.\\
%Data science approaches to high-dimensional analysis in Genomics, Inmunology and Epidemiology.\\
%Get a position in the field of Bioinformatics with emphasis in high-dimensional omics data analysis.\\
%Get a position in the field of Data Science with emphasis in public database analysis workflows.\\
% Get a position in the field of Genomics with emphasis in high-dimensional data analysis.\\
% Large-scale immunoassay data analysis of experiments with Public Health applications.\\
%aBASIC RESEARCH
%High-dimensional data analysis. Cell responses and plasticity to new inputs or environments.\\
% , plasticity and adaptability --> MALARIA PARASITE CELLS and IMMUNE CELLS
% obs INNER variability potential against EXTERNAL variability
% heterogeneity + environments % and its relevance to therapeutics
%aAPPLYED RESEARCH
%Bioinformatics. Optimization of therapy delivers and signals screening of Bioengineering solutions.\\
% processing % SynthBio approaches?
%Quantitative biology, Reproducible science, Free software solutions, and Open-source projects.\\

% microfluidics aims to test hypothesis in more controlled environments about environmnet-sensivite behaviours.
% environmentaly triggered phenotypic changes in highly dynamic organism
%discovery, execution and reporting methods
%Challenging experimental designs.
%Network biology approaches.
%Network biology for dynamic systems modeling.\\
%and interdisciplinary research. 

%\section{\sc Career goal}

%some KEY WORDS:
% emergent phenomena.
% gene regulation and 
% control and regulatory mechanisms for development 
% Bioengineer, theoretical and experimental
% analysis of high-dimensional data
% Applications to Genomics and Epidemiology
% Data-driven modeling
% System approaches to study multicellular developmental processes and organismal form.
% Theoretical
% Systems Biology applying Biophysics, Non-Linear Dynamics and Gene Regulatory Network modeling
% \textit{In silico} analysis of protein interactions and dynamics.

\section{\sc Education}

\begin{tabular}{ l l }
	2011-2015 & Universidad Nacional Mayor de San Marcos (UNMSM), Lima-Peru\\
	& {\bf BSc. Genetics and Biotechnology}\\
\end{tabular}\\

%Universidad Nacional Mayor de San Marcos (UNMSM), Lima-Peru \hfill 2011-2015\\%Mar Dec 
%%\vspace*{-.1in}
%{\bf BSc. Genetics and Biotechnology}\\%\hfill Apr 2016
%Average: 14.09. GPA equivalence\footnotemark\textsuperscript{,}\footnotemark: 3.05\\%2.79
%\footnotetext[1]{https://www.wes.org/gradeconversionguide/index.asp}
%\footnotetext[2]{http://bioegrad.berkeley.edu/prospectivegrads/gpaconversion}

%\begin{center}
%	\vspace{-9mm}
%	\begin{tabular}{lll}
%		\textit{Semester} & \textit{Relevant Subjects} & \textit{Grade} [0-20] \\
%		%		2012-I & General Biochemistry & 16\\
%		%2012-II & General Physics II & 16\\
%		%2013-I & Molecular Genetics & 16\\
%		2013-II & Bioinformatics & 16\\
%		%2013-II & Theoretical Ecology & 16\\
%		%		2014-I & General Systematics & 16\\
%		2014-II & Macromolecular Physics & 17\\
%		2014-II & Biomathematics & 18\\
%		%2015-II & Animal Biotechnology & 18\\
%	\end{tabular}
%	\vspace{1.8mm}
%\end{center}


\section{\sc Work History and affiliations}

\begin{tabular}{ l l l }
	2017- now & Universidad Peruana Cayetano Heredia (UPCH), Peru.&\\
	& Emergent Diseases and Climate Change Research Unit (EMERGE).&{\bf Internship}\\
	2016-2017 & Universidad Nacional de la Amazon\'ia Peruana (UNAP), Peru.&\\
	& Fundaci\'on para el Desarrollo Sostenible de la Amazon\'ia Baja.&{\bf Consultant}\\
	2015-2016 & U.S. Naval Medical Research Unit Six (NAMRU-6), Peru.&\\
	& Dept. of Parasitology, Div. of Immunology and Vaccine Development.&{\bf Thesis stud.}\\
	%2014-2015 & &{\bf Trainee}\\
	2012-2015 & Universidad Nacional Mayor de San Marcos (UNMSM), Peru.&\\
	&Laboratory of Physiology and Animal Reproduction (LFRA).&{\bf Researcher}\\
\end{tabular}\\

%Universidad Peruana Cayetano Heredia (UPCH), Peru.\\
%Emergent Diseases and Climate Change Research Unit (EMERGE).
%Public Health Faculty - %Facultad de Salud P\'ublica - Unidad de Enfermedades Emergentes\\
%{\bf Internship}.\\%Nov
%{\em Aim:} Risk factors and spatial cluster analysis from epidemiological surveys. \hfill 2017-now\\

%\vspace*{-.2in}
%\begin{list1}
%	\item[] {\bf Trainee} \hfill 2017- now\\%Nov Feb 
%	\vspace*{-.1in}
%	\begin{list2} %Job Description%
		%\item P.I.: Viviana Pinedo-Cancino, PhD. (UNAP)
		%\item Project: IgG antibodies as predictors of transmission and emergence of malaria.%%\\
%		\item Aim: Risk factors and spatial cluster analysis from epidemiological surveys.\\% by Dose-response analysis %Implemntation of a Reproducible workflow
		% Naturally acquired antibody profiles.
		%%% in Patients with Severe vivax malaria symptomatology!!!!
%	\end{list2}
%\end{list1}

%Universidad Nacional de la Amazon\'ia Peruana (UNAP), Peru.\\
%Fundaci\'on para el Desarrollo Sostenible de la Amazon\'ia Baja (Fundesab). %\\
%{\bf Consultant}.\\%Nov Feb 
%{\em Work:} ELISA plates standardization for large-scale serological surveillance. \hfill 2016-2017\\
%Facultad de Salud P\'ublica - Unidad de Enfermedades Emergentes\\

%\vspace*{-.2in}
%\begin{list1}
%	\item[] {\em Consultant} \hfill 2016-2017\\%Nov Feb 
%	\vspace*{-.1in}
%	\begin{list2} %Job Description%
		%\item P.I.: Viviana Pinedo-Cancino, PhD. (UNAP)
		%\item Project: IgG antibodies as predictors of transmission and emergence of malaria.%%\\
%		\item Work: ELISA plates standardization for large-scale serological surveillance.\\
		%\item Work: ELISA plates standardization for high-throughput sero-surveillance.\\%using R% by Dose-response analysis %Implemntation of a Reproducible workflow
		% Naturally acquired antibody profiles.
		%%% in Patients with Severe vivax malaria symptomatology!!!!
%	\end{list2}
%\end{list1}

%U.S. Naval Medical Research Unit Six (NAMRU-6), Peru.\\
%Dept. of Parasitology, Div. of Immunology and Vaccine Development. 
%{\bf Thesis student}.\\%Ago Dec 
%{\em Project:} Antibody response against \textit{Plasmodium vivax} using protein microarray. \hfill 2015-2016\\
%{\em Trainee experience:} Literature review meetings. Independent bioinformatic research.\hfill 2014-2015\\

%\vspace*{-.2in}
%\begin{list1}
%	\item[] {\em Thesis student} \hfill 2015-2016\\%Ago Dec 
%	\vspace*{-.1in}
%	\begin{list2} %Job Description%
		%\item Advisor: G. Christian Baldeviano, PhD.
%		\item Project: Antibody response against \textit{Plasmodium vivax} using protein microarray.\\
		%\item Project: Large screening of antibody response against \textit{Plasmodium vivax} malaria.\\
		%Project title: High-throughput Immunomics and Bioinformatics approach for the discovery of new antigenic determinants associated with protection against severe malaria\\
		% Naturally acquired antibody profiles.
		%%% in Patients with Severe vivax malaria symptomatology!!!!
%	\end{list2}
%%%%%%
%	\vspace*{-.1in}
%	\item[] {\em Trainee} \hfill 2014-2015\\%Jan Jul 
%	\vspace*{-.1in}
%	\begin{list2} %Job Description%
%		\item Experience: Literature review meetings. Independent research in bioinformatics.\\%Critical 
		%through applications in Biostatistics, Bioinformatics and \textit{in silico} Molecular Biology.
		%in \textit{Biostatistics} (microarray data analysis), 
		%\textit{Bioinformatics} (Image analysis of peptide arrays and phylogenetics), 
		%and \textit{Molecular Biology} (cloning and \textit{in silico} plasmid design). 
		%and \textit{Immunoassays} (DELI assay dose-response analysis of parasite growth against drug exposure).
		%colorimetric (...) for the detection of parasite growth biomarkers in response to drug
		% phylogenetics of selected candidates % cloning of candidates
%	\end{list2}
%\end{list1}

%{\bf Universidad Nacional Mayor de San Marcos (UNMSM)}, Peru.\\
%Laboratory of Physiology and Animal Reproduction\\
%\vspace*{-.1in}
%\begin{list1}
%	\item[] {\em Undergraduate researcher} \hfill 2012-2015\\%Mar Dec 
%	\vspace*{-.1in}
%	\begin{list2} %Job Description%
		%\item Advisor: Mg. Martha Valdivia Cuya
%		\item Experience: Grant proposal writing. Collaborative team-oriented laboratory research. %%of projects focusing in the reproductive problems of the Andean camelid Alpaca (\textit{Vicugna pacos}). %%
%		\item Main Project: In vitro effect of ELF-magnetic fields on sperm motility of Alpaca.\\ %%Biochemical evaluation of sperm reproductive quality after ELF-Magnetic Fields exposure.
%		%\item 2015 Project: Mitochondrial activity after sperm capacitation in Alpaca.
%		%\item 2014 Project: In vitro effect of ELF-Magnetic Fields on sperm motility of Alpaca.
%		%\item 2013 Project: Analysis of \textit{CatSper} gene expression in mice exposed to ELF-Magnetic Fields.
%		%\item 2012 Project: Cryopreservation of Spermatogonial Stem Cells of Alpaca.\\ %(\textit{Vicugna pacos})
%		%\item 2013 Project: Enzyme isolation and \textit{in vitro} evaluation of reproductive capacity in guinea pig (\textit{Cavia porcelus})\\
%	\end{list2}
%\end{list1}

\section{\sc Scientific Publications}

{\bf In preparation}

\begin{list1}
	%Antibody response against \textit{Plasmodium vivax} using protein microarray.
	\item[-] Pinedo-Cancino V., Baldeviano GC., Durand S., Saavedra-Langer R., Ventocilla JA., Arista KM., Arana A., Chasnamonte M., \textbf{Valle-Campos A.}, Smith ES., Ru\'iz-Mesia L.  ``Assessing malaria transmission intensity in a low endemic area of the Peruvian Amazon using parasitological and serological surveys".%\\%, to be submitted to {\em American Journal of Tropical Medicine and Hygene (AJTMH)}, Dec 2017.%\\%: A preliminary study
	\item[-] Saavedra-Langer R., Marapara J., \textbf{Valle-Campos A.}, Durand S., Chasnamote M., Silva H., Pinedo-Cancino V. ``IgG subclasses responses to excreted-secreted antigens of \textit{Plasmodium falciparum} in a low transmission malaria community of the Peruvian Amazon".%, to be submitted to {\em Acta Tropica}, Dec 2017.%\\%: A preliminary study
\end{list1}

\vspace*{-.1in}

{\bf Poster presentation}

\begin{list1}
	\item[-] Quispe J., \textbf{Valle-Campos A.}, Ulloa G., Rodriguez L., Granados E., Cruz V., Valdivia M, et al.\\ ``In vitro effect of Extremely Low Frequency Magnetic Field on the sperm motility of Alpacas", \\ {\em Annual Meeting of the Bioelectromagnetics Society - BioEM2015. Monterey, USA}, July 2015.\\%: A preliminary study %Li{\~n}an A., Limaymanta O., Granados E., Fuentes P., Carhuaricra D., 
	
	%Punil R., Murillo A., Carrasco M., Huaman A., Quispe J., Miranda J., Valladares K. and \textbf{Valle A}. ``Dermatoglyphic analysis on individuals with Down syndrome and Autism in comparison to a control group", {\em XV National Congress of Biology Students - CONEBIOL 2014. Lima, Peru}, October 2014
	
	%Valdivia M., Tataje L., Cisneros S., Carmen R., Guillen W. de los Santos, Davila D., \textbf{Valle A.}, et.al. ``Important genes for the camelid reproduction", {\em International Meeting of Research Groups in Basic and Applied Sciences - ASCILA 2012. Lima, Peru}, May 2012\\
\end{list1}

\section{\sc Research participation}

\begin{tabular}{ l c l }
	Risk factors and spatial cluster analysis from epidemiological surveys.&2017- now&Data analyst\\
	ELISA plates standardization for large-scale serological surveillance.&2016-2017&Data analyst\\
	Differential expression and data mining from protein microarray data.&2015-2016&Data analyst\\
	Dose-response assay of parasite growth against drug exposure (DELI).&2015-2016&Lab/Analyst\\
	Magnetic field exposure system and biochemical sperm evaluation.&2013-2014&Laboratory\\
	Gene cloning, protein expression, conventional and quantitative PCR.&2012-2013&Laboratory\\
	Stem-cell isolation, cryopreservation. Oocyte-sperm interaction.&2012-2013&Laboratory\\ 
	%  
	% to assess growth response against drug exposure
	%, DELI assay dose-response analysis of parasite growth against drug exposure
\end{tabular}\\

\section{\sc Financing obtained}

\begin{tabular}{ l l l }
	Mitochondrial activity after sperm capacitation in  Alpaca.&~~~~~~~2015&~~~~~S/1,500.00\\
	{\em In vitro} effect of Magnetic Fields on the sperm motility of Alpacas.&~~~~~~~2014&~~~~~S/1,500.00\\
%	Cryopreservation of Spermatogonial Stem Cells of Alpaca.&~~~~~~~2012&~~~~~S/1,500.00\\
\end{tabular}\\

%\section{\sc Data Science Skills}%Experienced in the Statistical computing software...
%\begin{tabular}{ l l }
%	{\bf Software}: &  R +packages:\\ %%some general-porpuse packages listed\\
%	{\bf Biostatistics}: & \texttt{limma} to test differential expression in DNA/protein microarray data.\\ %of genes
%	%\texttt{Hmisc} package %plus graphical summaries
%	{\bf Modeling}: & \texttt{drc} to iteratively fit 4-parameter log-logistic models to immunoassays.\\
%	{\bf Exploration}: & \texttt{tidyverse} for large dataset mining in genomics and epidemiology.\\% and \texttt{Hmisc}
%	{\bf Visualization}: & \texttt{base} and \texttt{ggplot2} graphics. \texttt{DiagrammeR} for graphs and flowcharts.\\ 
%	{\bf Reproducibility}: & \texttt{knitr} and \texttt{bookdown} to integrate text, code and results in reports.
%	%for reports with integrated
%	%immunoassays, drug-response assays, 
%	%Familiar with packages for Spatial data and Geographical information systems %layered grammar of graphics
%	%{\bf Frequently used}: & \texttt{knitr, Hmisc, XLConnect.}%\\
%\end{tabular}

\section{\sc Computational Skills}
\begin{tabular}{ l l}
	{\bf Programming Language}: & Python, Perl, Bash (Unix shell), SQL.\\
	%Applications: Numerical integration and Population dynamics simulations.\\
	{\bf Statistical package}: & R: \texttt{limma}, \texttt{drc}, \texttt{Hmisc}, \texttt{tidyverse}, \texttt{knitr}, \texttt{Diagrammer}.\\
	{\bf Bioinformatics}: & Arlequin, MrBayes, Artemis, VMD, PyMol, Ape.\\
	{\bf Operating System}: & GNU/Linux (Ubuntu distribution).\\ %%, Windows, OSX
	{\bf Document preparation}: & \LaTeX, R Markdown.\\ %SublimeText.%
	%using \texttt{knitr} package and Pandoc
	%%Word processor software  %, LibreOffice Writer, MS Word.\\	, LibreOffice, MS Office
	{\bf Version control}: & Git.%\\ %% ADD LINK!! %%  and repository
	%{\bf Image processing}: & \texttt{ImageJ} for relative quantification of synthetic peptide arrays.\\
	%{\bf Image processing and editing software}: & ImageJ, GIMP.\\ %Photoshop
	%Applications: Relative quantification of synthetic peptide arrays. Professional photo editing\\
	%{\bf Bioinformatics}: & Image analysis of peptide arrays using \texttt{ImageJ} and phylogenetics.\\
	%{\bf Bioinformatics}: & Applications to phylogenetics, genome assembly, and structural genomics.%\\
	%and genome assembly
	%software: (Arlequin, DnaSP, PhyML, MrBayes, BEAST, MEGA), (MODELLER, GROMACS, VMD, PyMol), (Artemis)
\end{tabular}\\

%\section{\sc Bioinformatic Skills}
%\begin{tabular}{l l}
%	{\bf Phylogenetics}: & Arlequin, DnaSP, PhyML, MrBayes, BEAST, MEGA.\\
%	{\bf Structural genomics}: & MODELLER, GROMACS, VMD, PyMol.\\
%	{\bf Genomics}: & Artemis, ACT, samtools, cuffdiff.\\
%	{\bf Cloning}: & ApE.%\\
%\end{tabular}


%\section{\sc Experimental Skills}
%%Experienced in 
%\begin{tabular}{ l l }
%	{\bf Genetics}: & Gene cloning, protein expression, conventional and quantitative PCR.\\
%	%genetic analysis techniques like %RealTime-
%	%Familiar with 
%	{\bf Biochemistry}: & Stem-cell isolation, cryopreservation, ELISA-based assays (e.g. DELI).\\ 
%	% Oocyte-sperm interaction 
%	% to assess growth response against drug exposure
%	%, DELI assay dose-response analysis of parasite growth against drug exposure
%	{\bf Optimization}: & Surface-response method with factorial design of experiments.\\
%\end{tabular}

%%%%%%%%%%%%%%%%%%%%%%%%%%%%%%%%%%%%%%%%%%%%%%%%%
\newpage

\section{\sc Personal Achievements}

{\bf Ranked 1\textsuperscript{st} at the UPCH X Course and Workshop on Molecular\\Biology Techniques Applied to Infectious and Tropical Diseases} \hfill Jan 2013\\
Highest grade among 40 graduate and undergraduate students. %in this intensive summer course.

{\bf Ranked 1\textsuperscript{st} at the UNMSM Public University Admission Test} \hfill Mar 2011\\
Highest score among the Basic Sciences Faculties from a total of 1000 applicants.\\%\vspace*{.05in}\\

\section{\sc Teaching experience}
%Invited teacher for the following third-year undergraduate elective courses: \\
%{\bf Biomathematics}:  {\em Population dynamics}. \hfill {December 2015}\\ Discrete modeling methods in Ecology. \\ %%Graph theory and Linear algebra 
{\bf Bioinformatics}: Data analysis of microarray based experiments. \hfill {2017-I}\\ Design, statistics and visualization using Bioconductor and Tidyverse in R. \\[4pt] 
{\bf Biomathematics}: Gene Regulatory Networks. \hfill {2015/16/17-I/II}\\ GRN topology and dynamics using Graph Theory and Finite Automata. \\[4pt] %% Graph theory to model Topology and Automata Networks for Dynamics of GRN %% Modeling of Networks
{\bf Horizontal Gene Transfer}: On \#tardigate and HGT bioinformatics. \hfill {2016-II}\\ 
Review of the controversy around the first tardigrade genome. \\


%\section{\sc Complementary activities}
%{\bf IT training}: CISCO-IT Essentials: PC Hardware and Assembly. UNI. Lima-Peru. \hfill {Jan 2015} \\
%%Universidad Nacional de Ingeniería 
%{\bf Sports}: UNMSM swimming team 2011-2013. 1\textsuperscript{st} place: 50m Fly at the National Sport Games 2012\\


%\section{\sc International training}
\section{\sc Complementary education}

{\bf CODATA-RDA School in Research Data Science}\hfill {Dec 2017}\\
	\vspace*{-.1in}%Selected with complete financial support
\begin{list2} %Job Description%
	\item Two weeks on database management, machine learning and infrastructure.%\textit{Experience:}
	\item ICTP-SAIFR. IFT-UNESP, S{\~a}o Paulo - Brazil. %International Center for Theoretical Physics - South American Institute for Fundamental Research (ICTP-SAIFR). IFT-UNESP, Sao Paulo- Brazil.%\textit{Organized by:} 
\end{list2}

{\bf Wellcome Genome Campus Advanced Courses:\\Working with Parasite Data Resources}\hfill {Oct 2016}\\
	\vspace*{-.1in}%Complete financial support
\begin{list2} %Job Description%
	\item One week on genomic, proteomic, metabolomic applications of \texttt{eupathdb.org}.
%	\item \textit{Venue:} Instituto del Higiene, Montevideo, Uruguay.%\\
	\item Wellcome Trust Sanger Institute, UK. Montevideo - Uruguay.%\\ %with EuPathDB and 
	%Resource Center the Eukaryotic Pathogen Bioinformatics Resource Center 
%	\item \textit{Time:} 35h
\end{list2}

{\bf Workshop EPONGE: Epidemiology meets Population Genetics}\hfill {Oct 2016}\\
	\vspace*{-.1in}
\begin{list2} %Job Description%
	\item One week on theory and update topics, including bayesian inference methods.
	\item University of Antwerp y UPCH. Lima - Peru.
	%\item \textit{Organized by:} Global Health Institute - University of Antwerp and UPCH %and UNAP.%\\
	%\item \textit{Time:} 30h
\end{list2}

%{\bf School on Physics Applications in Biology}\hfill {Jan 2016}\\
%	\vspace*{-.1in}%Selected with complete financial support
%\begin{list2} %Job Description%
%	\item Three weeks on game theory, non-linear dynamics and statistical physics.%\textit{Experience:}
%	\item ICTP-SAIFR. IFT-UNESP, S{\~a}o Paulo - Brazil. %International Center for Theoretical Physics - South American Institute for Fundamental Research (ICTP-SAIFR). IFT-UNESP, Sao Paulo- Brazil.%\textit{Organized by:} 
%\end{list2}

{\bf V Southern-Summer School on Mathematical Biology}\hfill {Jan 2016}\\
	\vspace*{-.1in}%Complete financial support
\begin{list2} %Job Description%
	\item One week on population dynamics modeling in ecology and epidemiology.% \textit{Experience:}
	\item ICTP-SAIFR. IFT-UNESP, S{\~a}o Paulo - Brazil.%\textit{Organized by:} 
\end{list2}

%{\bf Workshop on morphogenesis, models and evolution of \\developmental mechanisms}\hfill {Sep 2015}\\%http://c3.unam.mx/calendario/Externos/20150619141256220
%\vspace*{-.1in}%Complete financial support
%\begin{list2} %Job Description%
%	\item Two days of conferences directed by Stuart A. Newman.% \textit{Experience:}
%	\item Center for the Science of Complexity (C3). UNAM, Mexico City - Mexico.%\textit{Organized by:} 
%\end{list2}


%{\bf Minischool on Biophysics of Protein Interactions}\hfill {Mar 2015}\\
%\\and Onuchic Minicourse on Energy Landscapes
%	\vspace*{-.1in}%Selected with complete financial support
%\begin{list2} %Job Description%
%	\item One week on protein folding and electrostatic effects in biomolecules.%\textit{Experience:} 
%	\item ICTP-SAIFR. IFT-UNESP, S{\~a}o Paulo - Brazil.%\\%\textit{Organized by:} 
%\end{list2}

%{\bf Phylogenetics and Bioinformatics sequence analysis training - Level 1}\hfill {Jan 2015}\\
%	\vspace*{-.1in}
%\begin{list2} %Job Description%
%	\item \textit{Organized by:} U.S. Naval Medical Research Unit Six (NAMRU-6).%\\
%	%\item \textit{Time:} 30h
%\end{list2}

%{\bf Theory and software course: Gene Cloning}\hfill {Dec 2014}\\
%	\vspace*{-.1in}
%\begin{list2} %Job Description%
%	\item \textit{Venue:} Universidad Agraria La Molina.%\\
%%	\item \textit{Time:} 12h
%\end{list2}

%{\bf Course on genomic analysis of microorganism, sequencing, \\assemble and annotation}\hfill {Oct 2014}\\
%	\vspace*{-.1in}
%\begin{list2} %Job Description%
%	\item \textit{Venue:} Universidad Nacional Mayor de San Marcos.%\\
%%	\item \textit{Time:} 20h
%\end{list2}

{\bf Latin-American training workshop on molecular epidemiology \\applied to infectious diseases}\hfill {Nov 2013}\\
	\vspace*{-.1in}%Invited by the institution
\begin{list2} %Job Description%
	\item One week on genotyping and informatics for tuberculosis and malaria research. %leishmania, 
	\item ITM-Antwerp and IMTAvH-UPCH, Lima - Peru.\\% organaized by ITM-Antwerp (Belgium) and 
\end{list2}

%{\bf Basic Course-Workshop: Gene cloning and protein expression by \\recombinant DNA techniques}\hfill {Aug 2012}\\
%	\vspace*{-.1in}
%\begin{list2} %Job Description%
%	\item \textit{Organized by:} Lab. Bioinformatica y Biologia Molecular - UPCH.\\
%	%\textit{Venue:} Universidad Peruana Cayetano Heredia.\\
%	%\textit{Time:} 60h
%\end{list2}

\section{\sc Languages}
{\bf English}: Advance level in Reading, Speaking and Writing.\\
{\bf Spanish}: Advance level in Reading, Speaking and Writing. Mother tongue.\\
%{\bf English}: Advance Level complete. BRITANICO Institute. Lima-Peru. \hfill {2012-2014} \\%Set Feb 

\section{\sc References }

\begin{tabular}{ l l }
	G. Christian Baldeviano, PhD & Head, Immunology and Vaccine Development Unit \\
	Advisor & Naval Medical Research Unit Six (NAMRU-6)\\
	2014 - 2017 & gbaldevi@gmail.com\\
	&\\
	Viviana Pinedo Cancino, PhD, MSc & Research scientist and General Coordinator\\
	Supervisor & Universidad Nacional de la Amazon\'ia Peruana (UNAP)\\
	2016 - 2017 & vivi\_vane@hotmail.com\\
	%Andres G. Lescano, MSH, PhD & Professor and Coordinator, Masters' in Epidemiological Research\\
	%Advisor & Universidad Peruana Cayetano Heredia (UPCH)\\
	%Jan 2014 - Jun 2015 & andres.lescano.g@upch.pe\\
	&\\
	Prof. Walter Cabrera-Febola & Chief, Group of Natural Structures and Theoretical Research \\
	Teacher & Universidad Nacional Mayor de San Marcos (UNMSM)\\
	2013 - 2015 & febcawal@gmail.com\\	
\end{tabular}

%%%%%%%%%%%%%%%%%%%%%%%%%%%%%%%%%%%%%%%
%\newpage

%\section{\scshape APPENDIX }

%{\scshape \textbf{COMPLEMENTARY EDUCATION}}

%\section{\sc Computational courses}



%{\bf \textit{In silico} Workshop: Techniques on molecular modeling of proteins}\hfill {June 2015}\\
%\textit{Organized by:} Instituto Peruano de Genetica - IPEGEN\\
%\textit{Time:} 8h



%\section{\sc Experimental courses}



%{\bf X Course and Workshop on Molecular Biology Techniques Applied \\to Infectious and Tropical Diseases}\hfill {January 2013}\\
%Achievement: Ranked 1\textsuperscript{st}.\\
%\textit{Organized by:} IMTAvH, Unidad de Epidemiologia Molecular - UPCH.\\
%\textit{Venue:} Universidad Peruana Cayetano Heredia.\\
%\textit{Time:} 70h

%{\bf VI International campus course: Perspectives of reproductive \\technologies in the Andean Region}\hfill {January 2013}\\
%\textit{Organized by:} Vrije Universiteit Brussel (Belgium) and UNMSM.\\
%\textit{Time:} 30h\\




\end{resume}
\end{document}




