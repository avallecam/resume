\documentclass[margin,line]{res}


\oddsidemargin -.5in
\evensidemargin -.5in
\textwidth=6.0in
\itemsep=0in
\parsep=0in
\topmargin=0in
\topskip=0in
\textheight 10in
 
\newenvironment{list1}{
  \begin{list}{\ding{113}}{%
      \setlength{\itemsep}{0in}
      \setlength{\parsep}{0in} \setlength{\parskip}{0in}
      \setlength{\topsep}{0in} \setlength{\partopsep}{0in}
      \setlength{\leftmargin}{0.17in}}}{\end{list}}
\newenvironment{list2}{
  \begin{list}{$\bullet$}{%
      \setlength{\itemsep}{0in}
      \setlength{\parsep}{0in} \setlength{\parskip}{0in}
      \setlength{\topsep}{0in} \setlength{\partopsep}{0in}
      \setlength{\leftmargin}{0.2in}}}{\end{list}}
    
\begin{document}

\name{\LARGE Andree Valle Campos}
\address{Calle Tambo Huascar 201, San Miguel, Lima-Peru}
\address{\textit{contact:} avallecam@gmail.com or (+51)950951722}

\begin{resume}

\vspace*{.15in}

\section{\sc Research Interests}

Get a position in the field of Data Science for the Statistical analysis of high-dimensional data.\\
Applications to Genomics and Epidemiology following Reproducible Science principles.\\ 

%% Quantitative Biology. Data-driven modeling. Interdisciplinary research
%System approaches to study multicellular developmental processes and organismal form.\\%%  throughout development %%Theoretical and i

%Systems Biology applying Biophysics, Non-Linear Dynamics and Gene Regulatory Network modeling

%\section{\sc Interests}
%Get a position in the field of Genomics with emphasis in high-throughput experimental data analysis

%\textit{In silico} analysis of protein interactions and dynamics.

\section{\sc Education}
{\bf Universidad Nacional Mayor de San Marcos}, Lima-Peru \hfill Mar 2011 - Dec 2015\\
%\vspace*{-.1in}
BSc. Genetics and Biotechnology\\

\section{\sc Research Experience}
%%%%%%

{\bf U.S. Naval Medical Research Unit Six (NAMRU-6)}, Peru.\\
Department of Parasitology - Immunology and Vaccine Development Unit\\
\vspace*{-.1in}
\begin{list1}
	\item[] {\em Collaborator} \hfill {\bf Nov 2016 - Feb 2017}\\
	\vspace*{-.1in}
	\begin{list2} %Job Description%
		\item P.I.: Viviana Pinedo-Cancino, PhD. (UNAP)
		%\item Project: IgG antibodies as predictors of transmission and emergence of malaria.%%\\
		\item Work: A reproducible workflow for ELISA plates standardization by Dose-response analysis.\\
		% Naturally acquired antibody profiles.
		%%% in Patients with Severe vivax malaria symptomatology!!!!
	\end{list2}
%%%%%%%
	\item[] {\em Thesis Student} \hfill {\bf Ago 2015 - Dec 2016}\\
	\vspace*{-.1in}
	\begin{list2} %Job Description%
		\item Advisor: G. Christian Baldeviano, PhD.
		\item Project: Large screening of antibody profiles in response to vivax malaria infection.\\
		%Project title: High-throughput Immunomics and Bioinformatics approach for the discovery of new antigenic determinants associated with protection against severe malaria\\
		% Naturally acquired antibody profiles.
		%%% in Patients with Severe vivax malaria symptomatology!!!!
	\end{list2}
%%%%%%
	\vspace*{-.1in}
	\item[] {\em Trainee} \hfill {\bf Jan 2014 - Jul 2015}\\
	\vspace*{-.1in}
	\begin{list2} %Job Description%
		\item Experience in \textit{Biostatistics} (microarray data analysis), \textit{Bioinformatics} (Image analysis of peptide arrays and phylogenetics), \textit{Molecular Biology} (cloning and \textit{in silico} plasmid design). %%and \textit{Immunoassays} (DELI assay dose-response analysis of parasite growth against drug exposure).
		%colorimetric (...) for the detection of parasite growth biomarkers in response to drug
		% phylogenetics of selected candidates % cloning of candidates
	\end{list2}
\end{list1}

{\bf Universidad Nacional Mayor de San Marcos}, Lima, Peru.\\
Laboratory of Physiology and Animal Reproduction\\
\vspace*{-.1in}
\begin{list1}
	\item[] {\em Undergraduate researcher} \hfill {\bf Mar 2012 - Dec 2015}\\
	\vspace*{-.1in}
	\begin{list2} %Job Description%
		\item Advisor: Mg. Martha Valdivia Cuya
		\item Experience: Grant proposal writing and Research group leadership. %%of projects focusing in the reproductive problems of the Andean camelid Alpaca (\textit{Vicugna pacos}). %%
		\item Main Project: In vitro effect of ELF-Magnetic Fields on sperm motility of Alpaca.\\ %%Biochemical evaluation of sperm reproductive quality after ELF-Magnetic Fields exposure.
		%\item 2015 Project: Mithocondrial activity after sperm capacitation in Alpaca.
		%\item 2014 Project: In vitro effect of ELF-Magnetic Fields on sperm motility of Alpaca.
		%\item 2013 Project: Analysis of \textit{CatSper} gene expression in mice exposed to ELF-Magnetic Fields.
		%\item 2012 Project: Cryopreservation of Spermatogonial Stem Cells of Alpaca.\\ %(\textit{Vicugna pacos})
		%\item 2013 Project: Enzyme isolation and \textit{in vitro} evaluation of reproductive capacity in guinea pig (\textit{Cavia porcelus})\\
	\end{list2}
\end{list1}

\section{\sc Computational Skills}
{\bf Statistical computing software}: R\\ %%and Bioconductor packages
%Applications: Analysis and visualization of Microarray experiments and Immunoassay.\\ %GIS data
{\bf Programming Language}: Python, Bash (Unix shell).\\%Perl, 
%Applications: Numerical integration and Population dynamics simulations.\\
%{\bf Bioinformatic software}: Phylogenetics (Arlequin, DnaSP, PhyML, MrBayes, BEAST, MEGA). Structural Genomics (MODELLER, GROMACS, VMD, PyMol). Genome assembly (Artemis).\\
{\bf Version control system and repository}: Git, GitHub.\\ %% ADD LINK!! %%
{\bf Document preparation systems}: LaTeX, Markdown with Pandoc.\\ 
%%Word processor software  %, LibreOffice Writer, MS Word.\\
{\bf Image processing and editing software}: ImageJ, GIMP, Photoshop.\\
%Applications: Relative quantification of synthetic peptide arrays. Professional photo editing\\
{\bf Operating System}: GNU/Linux (Ubuntu distribution).\\ %%, Windows, OSX

\section{\sc Data Science Skills}%Experienced in the ...
{\bf Biostatistics}: Bioconductor package \texttt{limma} for analysis of Genomics data, and \texttt{Hmisc} package for inferential statistics summary tables and visualization.\\ 
{\bf Data manipulation}: Core \texttt{tidyverse} packages for data import, management and modeling, including regular expression to edit raw strings and factor variables.\\
{\bf Data visualization}: Grammar of Graphics using \texttt{ggplot2} and \texttt{base} packages. Familiar with packages for Spatial data and Geographical information systems.\\ %immunoassays, drug-response assays, 


\section{\sc Experimental Skills}
{\bf Molecular Biology}: Experienced in gene cloning, protein expression, and genetic analysis techniques like conventional PCR and quantitative RealTime-PCR.\\
{\bf Biochemistry}: Familiar with stem-cell isolation, characterization, cryopreservation, oocyte-sperm interaction assays and spermiogram test.\\ %immunoassays, drug-response assays, 

%%%%%%%%%%%%%%%%%%%%%%%%%%%%%%%%%%%%%%%%%%%%%%%%%
\newpage

\section{\sc Personal Achievements}

{\bf Ranked 1\textsuperscript{st} at the X Course and Workshop on Molecular Biology \\Techniques Applied to Infectious and Tropical Diseases} \hfill January 2013\\
Highest grade among 40 graduate and undergraduate students on the intensive summer course organized by the IMTAvH - UPCH.

{\bf Ranked 1\textsuperscript{st} at the UNMSM Public University Admission Test} \hfill March 2011\\
Highest score among the Basic Sciences Faculties from a total of 1000 applicants.\\%\vspace*{.05in}\\

%\section{\sc International training}
\section{\sc Complementary education}
%{\bf School on Physics Applications in Biology}\hfill {January 2016}\\
%Selected with complete financial support
%\begin{list2} %Job Description%
%	\item \textit{Experience:} Three-week school on Game theory, Non-linear dynamics and Statistical physics.
%	\item \textit{Organized by:} ICTP-SAIFR. IFT-UNESP, Sao Paulo - Brazil. %International Center for Theoretical Physics - South American Institute for Fundamental Research (ICTP-SAIFR). IFT-UNESP, Sao Paulo- Brazil.
%\end{list2}

{\bf Wellcome Genome Campus Advanced Courses:\\Working with Parasite Data Resources}\hfill {October 2016}\\
	\vspace*{-.1in}%Complete financial support
\begin{list2} %Job Description%
	\item \textit{Venue:} Instituto del Higiene, Montevideo, Uruguay.%\\
	\item \textit{Organized by:} EuPathDB and Wellcome Trust Sanger Institute, UK. %\\ Resource Center
%the Eukaryotic Pathogen Bioinformatics Resource Center 
%	\item \textit{Time:} 35h
\end{list2}

{\bf Workshop EPONGE: Epidemiology meets POpulation GEnetics}\hfill {October 2016}\\
	\vspace*{-.1in}
\begin{list2} %Job Description%
	\item \textit{Organized by:} Global Health Institute - University of Antwerp and UPCH %and UNAP.%\\
	%\item \textit{Time:} 30h
\end{list2}

{\bf V Southern-Summer School on Mathematical Biology}\hfill {January 2016}\\
	\vspace*{-.1in}%Complete financial support
\begin{list2} %Job Description%
	\item \textit{Experience:} One-week school on Population dynamics modeling.
	\item \textit{Organized by:} ICTP-SAIFR. IFT-UNESP, Sao Paulo - Brazil.
\end{list2}

{\bf Phylogenetics and Bioinformatics sequence analysis training - Level 1}\hfill {January 2015}\\
	\vspace*{-.1in}
\begin{list2} %Job Description%
	\item \textit{Organized by:} U.S. Naval Medical Research Unit Six (NAMRU-6).%\\
	%\item \textit{Time:} 30h
\end{list2}


%{\bf Minischool on Biophysics of Protein Interactions \\and Onuchic Minicourse on Energy Landscapes}\hfill {March 2015}\\
%Selected with complete financial support
%\begin{list2} %Job Description%
%	\item \textit{Experience:} One-week school on protein folding and electrostatic effects in biomolecules.
%	\item \textit{Organized by:} ICTP-SAIFR. IFT-UNESP, Sao Paulo - Brazil.
%\end{list2}

%{\bf Theory and software course: Gene Cloning}\hfill {December 2014}%\\
%\begin{list2} %Job Description%
%	\item \textit{Venue:} Universidad Agraria La Molina.%\\
%\textit{Time:} 12h
%\end{list2}


%{\bf Course on genomic analysis of microorganism, sequencing, \\assemble and annotation}\hfill {October 2014}%\\
%\begin{list2} %Job Description%
%	\item \textit{Venue:} Universidad Nacional Mayor de San Marcos.%\\
%\textit{Time:} 20h
%\end{list2}


{\bf Latin-American training workshop on molecular epidemiology \\applied to infectious diseases}\hfill {November 2013}\\
	\vspace*{-.1in}%Invited by the institution
\begin{list2} %Job Description%
	\item \textit{Experience:} Genotyping and informatics for tuberculosis and malaria research. %leishmania, 
	\item \textit{Organized by:} ITM Antwerp (Belgium) and IMTAvH (Peru), Lima - Peru.\\
\end{list2}



\section{\sc Poster presentation}

Quispe J., \textbf{Valle-Campos A.}, Ulloa G., Rodriguez L., Liñan A., Limaymanta O., Granados E., Fuentes P., Carhuaricra D., Cruz V. and Valdivia M.\\ ``In vitro effect of Extremely Low Frequency Magnetic Field on the sperm motility of Alpacas: A preliminary study", \\ {\em Annual Meeting of the Bioelectromagnetics Society - BioEM2015. Monterey, USA}, July 2015\\

%Punil R., Murillo A., Carrasco M., Huaman A., Quispe J., Miranda J., Valladares K. and \textbf{Valle A}. ``Dermatoglyphic analysis on individuals with Down syndrome and Autism in comparison to a control group", {\em XV National Congress of Biology Students - CONEBIOL 2014. Lima, Peru}, October 2014

%Valdivia M., Tataje L., Cisneros S., Carmen R., Guillen W. de los Santos, Davila D., \textbf{Valle A.}, et.al. ``Important genes for the camelid reproduction", {\em International Meeting of Research Groups in Basic and Applied Sciences - ASCILA 2012. Lima, Peru}, May 2012\\


\section{\sc Teaching experience}
%Invited teacher for the following third-year undergraduate elective courses: \\
%{\bf Biomathematics}:  {\em Population dynamics}. \hfill {December 2015}\\ Discrete modeling methods in Ecology. \\ %%Graph theory and Linear algebra 
{\bf Biomathematics}: {\em Gene Regulatory Networks}. \hfill {December 2015/16}\\ GRN topology and dynamics using Graph Theory and Finite Automata. \\ %% Graph theory to model Topology and Automata Networks for Dynamics of GRN %% Modeling of Networks
{\bf Horizontal Gene Transfer}: {\em On \#tardigate and HGT bioinformatics}. \hfill {December 2016}\\ Review of the controversy on extensive HGT from the first tardigrade genome. \\

\section{\sc Languages}
{\bf English}: {\em Advance Level complete}. Instituto BRITÁNICO. \hfill {Set 2012 - Feb 2014} \\

\section{\sc References }

\begin{tabular}{ l l }
	G. Christian Baldeviano, PhD & Head, Immunology and Vaccine Development Unit \\
	Advisor & Naval Medical Research Unit Six (NAMRU-6)\\
	Jan 2014 - present & geralc.baldeviano.fn@mail.mil\\
	&\\
	Andres G. Lescano, MSH, PhD & Associate Professor and Masters' Program Coordinator\\
	Advisor & Universidad Peruana Cayetano Heredia\\
	Jan 2014 - Jun 2015 & andres.lescano.g@upch.pe\\
	&\\
	Prof. Walter Cabrera-Febola & Chief, Group of Natural Structures and Theoretical Research \\
	Teacher & Universidad Nacional Mayor de San Marcos\\
	Mar 2013 - present & febcawal@gmail.com\\	
\end{tabular}

%%%%%%%%%%%%%%%%%%%%%%%%%%%%%%%%%%%%%%%
%\newpage

%\section{\scshape APPENDIX }

%{\scshape \textbf{COMPLEMENTARY EDUCATION}}

%\section{\sc Computational courses}



%{\bf \textit{In silico} Workshop: Techniques on molecular modeling of proteins}\hfill {June 2015}\\
%\textit{Organized by:} Instituto Peruano de Genetica - IPEGEN\\
%\textit{Time:} 8h








%\section{\sc Experimental courses}

%{\bf X Course and Workshop on Molecular Biology Techniques Applied \\to Infectious and Tropical Diseases}\hfill {January 2013}\\
%Achievement: Ranked 1\textsuperscript{st}.\\
%\textit{Organized by:} IMTAvH, Unidad de Epidemiologia Molecular - UPCH.\\
%\textit{Venue:} Universidad Peruana Cayetano Heredia.\\
%\textit{Time:} 70h

%{\bf VI International campus course: Perspectives of reproductive \\technologies in the Andean Region}\hfill {January 2013}\\
%\textit{Organized by:} Vrije Universiteit Brussel (Belgium) and UNMSM.\\
%\textit{Time:} 30h\\

%{\bf Basic Course-Workshop: Gene cloning and protein expression by \\recombinant DNA techniques}\hfill {August 2012}\\
%\textit{Organized by:} Lab. Bioinformatica y Biologia Molecular - UPCH.\\
%\textit{Venue:} Universidad Peruana Cayetano Heredia.\\
%\textit{Time:} 60h




\end{resume}
\end{document}




