\documentclass[margin,line]{res}

\oddsidemargin -.5in
\evensidemargin -.5in
\textwidth=6.0in
\itemsep=0in
\parsep=0in
\topmargin=0in
\topskip=0in
\textheight 10in
 
\newenvironment{list1}{
  \begin{list}{\ding{113}}{%
      \setlength{\itemsep}{0in}
      \setlength{\parsep}{0in} \setlength{\parskip}{0in}
      \setlength{\topsep}{0in} \setlength{\partopsep}{0in}
      \setlength{\leftmargin}{0.17in}}}{\end{list}}
\newenvironment{list2}{
  \begin{list}{$\bullet$}{%
      \setlength{\itemsep}{0in}
      \setlength{\parsep}{0in} \setlength{\parskip}{0in}
      \setlength{\topsep}{0in} \setlength{\partopsep}{0in}
      \setlength{\leftmargin}{0.2in}}}{\end{list}}
    
\begin{document}

\name{\LARGE Andree Valle Campos}
\address{Calle Tambo Huascar 201, San Miguel, Lima-Peru}
\address{\textit{contact:} avallecam@gmail.com or (+51)950951722}

\begin{resume}

\vspace*{.15in}

\section{\sc Research Interests}% 

%Get a position in the field of Genomics with emphasis in high-dimensional data analysis

Quantitative biology. Statistical data analysis following Reproducible Science principles.\\Network biology approaches. Applications to Genomics, Immunology and Epidemiology.%\\
%Network biology for dynamic systems modeling.\\%and interdisciplinary research. 

%\section{\sc Career goal}

%some KEY WORDS:
% emergent phenomena.
% gene regulation and 
% control and regulatory mechanisms for development 
% Bioengineer, theoretical and experimental
% analysis of high-dimensional data
% Applications to Genomics and Epidemiology
% Data-driven modeling
% System approaches to study multicellular developmental processes and organismal form.
% Theoretical
% Systems Biology applying Biophysics, Non-Linear Dynamics and Gene Regulatory Network modeling
% \textit{In silico} analysis of protein interactions and dynamics.

\section{\sc Education}
{\bf Universidad Nacional Mayor de San Marcos}, Lima-Peru \hfill Mar 2011 - Dec 2015\\
%\vspace*{-.1in}
BSc. Genetics and Biotechnology\hfill Jun 2016\\
Average: 14.09. GPA equivalence\footnotemark\textsuperscript{,}\footnotemark: 3.05\\%2.79
\footnotetext[1]{https://www.wes.org/gradeconversionguide/index.asp}
\footnotetext[2]{http://bioegrad.berkeley.edu/prospectivegrads/gpaconversion}

\begin{center}
	\vspace{-9mm}
	\begin{tabular}{lll}
		\textit{Semester} & \textit{Relevant Subjects} & \textit{Grade} [0-20] \\
		%		2012-I & General Biochemistry & 16\\
		%2012-II & General Physics II & 16\\
		%2013-I & Molecular Genetics & 16\\
		2013-II & Bioinformatics & 16\\
		%2013-II & Theoretical Ecology & 16\\
		%		2014-I & General Systematics & 16\\
		2014-II & Macromolecular Physics & 17\\
		2014-II & Biomathematics & 18\\
		%2015-II & Animal Biotechnology & 18\\
	\end{tabular}
	\vspace{1.8mm}
\end{center}


\section{\sc Research Experience}

{\bf U.S. Naval Medical Research Unit Six (NAMRU-6)}, Peru.\\
Department of Parasitology - Immunology and Vaccine Development Unit\\
\vspace*{-.1in}
\begin{list1}
	\item[] {\em Consultant} \hfill {\bf Nov 2016 - Feb 2017}\\
	\vspace*{-.1in}
	\begin{list2} %Job Description%
		%\item P.I.: Viviana Pinedo-Cancino, PhD. (UNAP)
		%\item Project: IgG antibodies as predictors of transmission and emergence of malaria.%%\\
		\item Work: ELISA plates standardization for high-throughput sero-surveillance using R.\\% by Dose-response analysis %Implemntation of a Reproducible workflow
		% Naturally acquired antibody profiles.
		%%% in Patients with Severe vivax malaria symptomatology!!!!
	\end{list2}
%%%%%%%
	\vspace*{-.1in}
	\item[] {\em Thesis Student} \hfill {\bf Ago 2015 - Dec 2016}\\
	\vspace*{-.1in}
	\begin{list2} %Job Description%
		%\item Advisor: G. Christian Baldeviano, PhD.
		\item Project: Large screening of antibody profiles in response to vivax malaria infection.\\
		%Project title: High-throughput Immunomics and Bioinformatics approach for the discovery of new antigenic determinants associated with protection against severe malaria\\
		% Naturally acquired antibody profiles.
		%%% in Patients with Severe vivax malaria symptomatology!!!!
	\end{list2}
%%%%%%
	\vspace*{-.1in}
	\item[] {\em Trainee} \hfill {\bf Jan 2014 - Jul 2015}\\
	\vspace*{-.1in}
	\begin{list2} %Job Description%
		\item Experience: Critical review of literature meetings. Independent computational research.
		%through applications in Biostatistics, Bioinformatics and \textit{in silico} Molecular Biology.
		%in \textit{Biostatistics} (microarray data analysis), 
		%\textit{Bioinformatics} (Image analysis of peptide arrays and phylogenetics), 
		%and \textit{Molecular Biology} (cloning and \textit{in silico} plasmid design). 
		%and \textit{Immunoassays} (DELI assay dose-response analysis of parasite growth against drug exposure).
		%colorimetric (...) for the detection of parasite growth biomarkers in response to drug
		% phylogenetics of selected candidates % cloning of candidates
	\end{list2}
\end{list1}

{\bf Universidad Nacional Mayor de San Marcos (UNMSM)}, Peru.\\
Laboratory of Physiology and Animal Reproduction\\
\vspace*{-.1in}
\begin{list1}
	\item[] {\em Undergraduate researcher} \hfill {\bf Mar 2012 - Dec 2015}\\
	\vspace*{-.1in}
	\begin{list2} %Job Description%
		%\item Advisor: Mg. Martha Valdivia Cuya
		\item Experience: Grant proposal writing. Collaborative team-oriented laboratory research. %%of projects focusing in the reproductive problems of the Andean camelid Alpaca (\textit{Vicugna pacos}). %%
		\item Main Project: In vitro effect of ELF-magnetic fields on sperm motility of Alpaca.\\ %%Biochemical evaluation of sperm reproductive quality after ELF-Magnetic Fields exposure.
		%\item 2015 Project: Mithocondrial activity after sperm capacitation in Alpaca.
		%\item 2014 Project: In vitro effect of ELF-Magnetic Fields on sperm motility of Alpaca.
		%\item 2013 Project: Analysis of \textit{CatSper} gene expression in mice exposed to ELF-Magnetic Fields.
		%\item 2012 Project: Cryopreservation of Spermatogonial Stem Cells of Alpaca.\\ %(\textit{Vicugna pacos})
		%\item 2013 Project: Enzyme isolation and \textit{in vitro} evaluation of reproductive capacity in guinea pig (\textit{Cavia porcelus})\\
	\end{list2}
\end{list1}

\section{\sc Data Science Skills}%Experienced in the Statistical computing software...
\begin{tabular}{ l l }
	{\bf Software}: &  R + packages:\\ %%some general-porpuse packages listed\\
	{\bf Biostatistics}: & \texttt{limma} for differential expression of genes in microarray data.\\ 
	%\texttt{Hmisc} package %plus graphical summaries
	{\bf Modeling}: & \texttt{drc} to iteratively fit 4-par log-logistic models to immunoassays.\\
	{\bf Exploration}: & \texttt{tidyverse} and \texttt{Hmisc} for large dataset analysis in epidemiology.\\
	%pipe operators (\texttt{\%>\%}) and regular expressions.\\ %to edit raw strings and factor variables
	%> tidyverse::tidyverse_packages()
	%[1] "broom"     "dplyr"     "forcats"   "ggplot2"   "haven"     "httr"     
	%[7] "hms"       "jsonlite"  "lubridate" "magrittr"  "modelr"    "purrr"    
	%[13] "readr"     "readxl"    "stringr"   "tibble"    "rvest"     "tidyr"    
	%[19] "xml2"      "tidyverse"
	{\bf Visualization}: & \texttt{base} and \texttt{ggplot2} graphics. \texttt{DiagrammeR} for graphs and flowcharts.\\ 
	{\bf Reproducibility}: & \texttt{knitr} and \texttt{bookdown} for reports integrating text, code and results.
	%immunoassays, drug-response assays, 
	%Familiar with packages for Spatial data and Geographical information systems %layered grammar of graphics
	%{\bf Frequently used}: & \texttt{knitr, Hmisc, XLConnect.}%\\
\end{tabular}


\section{\sc Bioinformatic Softwares}
\begin{tabular}{l l}
	{\bf Phylogenetics}: & Arlequin, DnaSP, PhyML, MrBayes, BEAST, MEGA.\\
	{\bf Structural genomics}: & MODELLER, GROMACS, VMD, PyMol.\\
	{\bf Genomics}: & Artemis, ACT, samtools, cuffdiff.\\
	{\bf Cloning}: & ApE.%\\
\end{tabular}

\section{\sc Computational Skills}
\begin{tabular}{ l l}
	{\bf Programming Language}: & Python, Perl, Bash (Unix shell).\\
	%Applications: Numerical integration and Population dynamics simulations.\\
	{\bf Operating System}: & GNU/Linux (Ubuntu distribution).\\ %%, Windows, OSX
	{\bf Version control}: & Git.\\ %% ADD LINK!! %%  and repository
	%{\bf Image processing}: & \texttt{ImageJ} for relative quantification of synthetic peptide arrays.\\
	%{\bf Image processing and editing software}: & ImageJ, GIMP.\\ %Photoshop
	%Applications: Relative quantification of synthetic peptide arrays. Professional photo editing\\
	%{\bf Bioinformatics}: & Image analysis of peptide arrays using \texttt{ImageJ} and phylogenetics.\\
	%{\bf Bioinformatics}: & Applications to phylogenetics, genome assembly, and structural genomics.%\\
	%and genome assembly
	%software: (Arlequin, DnaSP, PhyML, MrBayes, BEAST, MEGA), (MODELLER, GROMACS, VMD, PyMol), (Artemis)
	{\bf Document preparation}: & \LaTeX, R Markdown. SublimeText as editor.%\\ 
	%using \texttt{knitr} package and Pandoc
	%%Word processor software  %, LibreOffice Writer, MS Word.\\
\end{tabular}

\section{\sc Experimental Skills}
%Experienced in 
\begin{tabular}{ l l }
	{\bf Genetics}: & Gene cloning, protein expression, conventional and quantitative PCR.\\
	%genetic analysis techniques like %RealTime-
	%Familiar with 
	{\bf Biochemistry}: & Stem-cell isolation, cryopreservation, ELISA-based assays (e.g. DELI).\\ 
	% Oocyte-sperm interaction 
	% to assess growth response against drug exposure
	%, DELI assay dose-response analysis of parasite growth against drug exposure
	{\bf Optimization}: & Surface-response method with factorial design of experiments.\\
\end{tabular}

%%%%%%%%%%%%%%%%%%%%%%%%%%%%%%%%%%%%%%%%%%%%%%%%%
\newpage

\section{\sc Personal Achievements}

{\bf Ranked 1\textsuperscript{st} at the UPCH X Course and Workshop on Molecular\\Biology Techniques Applied to Infectious and Tropical Diseases} \hfill January 2013\\
Highest grade among 40 graduate and undergraduate students. %in this intensive summer course.

{\bf Ranked 1\textsuperscript{st} at the UNMSM Public University Admission Test} \hfill March 2011\\
Highest score among the Basic Sciences Faculties from a total of 1000 applicants.\\%\vspace*{.05in}\\

%\section{\sc International training}
\section{\sc Complementary education}

{\bf Wellcome Genome Campus Advanced Courses:\\Working with Parasite Data Resources}\hfill {October 2016}\\
	\vspace*{-.1in}%Complete financial support
\begin{list2} %Job Description%
	\item One week on genomics, proteomics, metabolomics applications of \texttt{eupathdb.org}.
%	\item \textit{Venue:} Instituto del Higiene, Montevideo, Uruguay.%\\
	\item Wellcome Trust Sanger Institute, UK. Montevideo - Uruguay.%\\ %with EuPathDB and 
	%Resource Center the Eukaryotic Pathogen Bioinformatics Resource Center 
%	\item \textit{Time:} 35h
\end{list2}

{\bf Workshop EPONGE: Epidemiology meets Population Genetics}\hfill {October 2016}\\
	\vspace*{-.1in}
\begin{list2} %Job Description%
	\item One week on theory and update topics, including bayesian inference methods.
	\item University of Antwerp y UPCH. Lima - Peru.
	%\item \textit{Organized by:} Global Health Institute - University of Antwerp and UPCH %and UNAP.%\\
	%\item \textit{Time:} 30h
\end{list2}

{\bf School on Physics Applications in Biology}\hfill {January 2016}\\
	\vspace*{-.1in}%Selected with complete financial support
\begin{list2} %Job Description%
	\item Three weeks on game theory, non-linear dynamics and statistical physics.%\textit{Experience:}
	\item ICTP-SAIFR. IFT-UNESP, S{\~a}o Paulo - Brazil. %International Center for Theoretical Physics - South American Institute for Fundamental Research (ICTP-SAIFR). IFT-UNESP, Sao Paulo- Brazil.%\textit{Organized by:} 
\end{list2}

{\bf V Southern-Summer School on Mathematical Biology}\hfill {January 2016}\\
	\vspace*{-.1in}%Complete financial support
\begin{list2} %Job Description%
	\item One week on population dynamics modeling in ecology and epidemiology.% \textit{Experience:}
	\item ICTP-SAIFR. IFT-UNESP, S{\~a}o Paulo - Brazil.%\textit{Organized by:} 
\end{list2}

%{\bf Workshop on morphogenesis, models and evolution of \\developmental mechanisms}\hfill {September 2015}\\%http://c3.unam.mx/calendario/Externos/20150619141256220
%\vspace*{-.1in}%Complete financial support
%\begin{list2} %Job Description%
%	\item Two days of conferences directed by Stuart A. Newman.% \textit{Experience:}
%	\item Center for the Science of Complexity (C3). UNAM, Mexico City - Mexico.%\textit{Organized by:} 
%\end{list2}


%{\bf Minischool on Biophysics of Protein Interactions}\hfill {March 2015}\\
%\\and Onuchic Minicourse on Energy Landscapes
%	\vspace*{-.1in}%Selected with complete financial support
%\begin{list2} %Job Description%
%	\item One week on protein folding and electrostatic effects in biomolecules.%\textit{Experience:} 
%	\item ICTP-SAIFR. IFT-UNESP, S{\~a}o Paulo - Brazil.%\\%\textit{Organized by:} 
%\end{list2}

%{\bf Phylogenetics and Bioinformatics sequence analysis training - Level 1}\hfill {January 2015}\\
%	\vspace*{-.1in}
%\begin{list2} %Job Description%
%	\item \textit{Organized by:} U.S. Naval Medical Research Unit Six (NAMRU-6).%\\
%	%\item \textit{Time:} 30h
%\end{list2}

%{\bf Theory and software course: Gene Cloning}\hfill {December 2014}\\
%	\vspace*{-.1in}
%\begin{list2} %Job Description%
%	\item \textit{Venue:} Universidad Agraria La Molina.%\\
%%	\item \textit{Time:} 12h
%\end{list2}

%{\bf Course on genomic analysis of microorganism, sequencing, \\assemble and annotation}\hfill {October 2014}\\
%	\vspace*{-.1in}
%\begin{list2} %Job Description%
%	\item \textit{Venue:} Universidad Nacional Mayor de San Marcos.%\\
%%	\item \textit{Time:} 20h
%\end{list2}

{\bf Latin-American training workshop on molecular epidemiology \\applied to infectious diseases}\hfill {November 2013}\\
	\vspace*{-.1in}%Invited by the institution
\begin{list2} %Job Description%
	\item One week on genotyping and informatics for tuberculosis and malaria research. %leishmania, 
	\item ITM-Antwerp and IMTAvH-UPCH, Lima - Peru.\\% organaized by ITM-Antwerp (Belgium) and 
\end{list2}

%{\bf Basic Course-Workshop: Gene cloning and protein expression by \\recombinant DNA techniques}\hfill {August 2012}\\
%	\vspace*{-.1in}
%\begin{list2} %Job Description%
%	\item \textit{Organized by:} Lab. Bioinformatica y Biologia Molecular - UPCH.\\
%	%\textit{Venue:} Universidad Peruana Cayetano Heredia.\\
%	%\textit{Time:} 60h
%\end{list2}



\section{\sc Poster presentation}

Quispe J., \textbf{Valle-Campos A.}, Ulloa G., Rodriguez L., Granados E., Cruz V., Valdivia M, et al.\\ ``In vitro effect of Extremely Low Frequency Magnetic Field on the sperm motility of Alpacas", \\ {\em Annual Meeting of the Bioelectromagnetics Society - BioEM2015. Monterey, USA}, July 2015.\\%: A preliminary study
%Li{\~n}an A., Limaymanta O., Granados E., Fuentes P., Carhuaricra D., 

%Punil R., Murillo A., Carrasco M., Huaman A., Quispe J., Miranda J., Valladares K. and \textbf{Valle A}. ``Dermatoglyphic analysis on individuals with Down syndrome and Autism in comparison to a control group", {\em XV National Congress of Biology Students - CONEBIOL 2014. Lima, Peru}, October 2014

%Valdivia M., Tataje L., Cisneros S., Carmen R., Guillen W. de los Santos, Davila D., \textbf{Valle A.}, et.al. ``Important genes for the camelid reproduction", {\em International Meeting of Research Groups in Basic and Applied Sciences - ASCILA 2012. Lima, Peru}, May 2012\\

\section{\sc Teaching experience}
%Invited teacher for the following third-year undergraduate elective courses: \\
%{\bf Biomathematics}:  {\em Population dynamics}. \hfill {December 2015}\\ Discrete modeling methods in Ecology. \\ %%Graph theory and Linear algebra 
{\bf Biomathematics}: Gene Regulatory Networks. \hfill {December 2015/16}\\ GRN topology and dynamics using Graph Theory and Finite Automata. \\[4pt] %% Graph theory to model Topology and Automata Networks for Dynamics of GRN %% Modeling of Networks
{\bf Horizontal Gene Transfer}: On \#tardigate and HGT bioinformatics. \hfill {December 2016}\\ 
Review of the controversy around the first tardigrade genome. \\


\section{\sc Languages}
{\bf English}: {\em Advance Level complete}. BRITANICO Institute. Lima-Peru. \hfill {Set 2012 - Feb 2014} \\

\section{\sc References }

\begin{tabular}{ l l }
	G. Christian Baldeviano, PhD & Head, Immunology and Vaccine Development Unit \\
	Advisor & Naval Medical Research Unit Six (NAMRU-6)\\
	Jan 2014 - present & geralc.baldeviano.fn@mail.mil\\
	&\\
	Andres G. Lescano, MSH, PhD & Professor and Coordinator, Masters' in Epidemiological Research\\
	Advisor & Universidad Peruana Cayetano Heredia (UPCH)\\
	Jan 2014 - Jun 2015 & andres.lescano.g@upch.pe\\
	&\\
	Prof. Walter Cabrera-Febola & Chief, Group of Natural Structures and Theoretical Research \\
	Teacher & Universidad Nacional Mayor de San Marcos (UNMSM)\\
	Mar 2013 - Dec 2015 & febcawal@gmail.com\\	
\end{tabular}

%%%%%%%%%%%%%%%%%%%%%%%%%%%%%%%%%%%%%%%
%\newpage

%\section{\scshape APPENDIX }

%{\scshape \textbf{COMPLEMENTARY EDUCATION}}

%\section{\sc Computational courses}



%{\bf \textit{In silico} Workshop: Techniques on molecular modeling of proteins}\hfill {June 2015}\\
%\textit{Organized by:} Instituto Peruano de Genetica - IPEGEN\\
%\textit{Time:} 8h



%\section{\sc Experimental courses}



%{\bf X Course and Workshop on Molecular Biology Techniques Applied \\to Infectious and Tropical Diseases}\hfill {January 2013}\\
%Achievement: Ranked 1\textsuperscript{st}.\\
%\textit{Organized by:} IMTAvH, Unidad de Epidemiologia Molecular - UPCH.\\
%\textit{Venue:} Universidad Peruana Cayetano Heredia.\\
%\textit{Time:} 70h

%{\bf VI International campus course: Perspectives of reproductive \\technologies in the Andean Region}\hfill {January 2013}\\
%\textit{Organized by:} Vrije Universiteit Brussel (Belgium) and UNMSM.\\
%\textit{Time:} 30h\\




\end{resume}
\end{document}




