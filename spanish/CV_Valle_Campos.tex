\documentclass[margin,line]{res}


\oddsidemargin -.5in
\evensidemargin -.5in
\textwidth=6.0in
\itemsep=0in
\parsep=0in
\topmargin=0in
\topskip=0in
\textheight 10in
 
\newenvironment{list1}{
  \begin{list}{\ding{113}}{%
      \setlength{\itemsep}{0in}
      \setlength{\parsep}{0in} \setlength{\parskip}{0in}
      \setlength{\topsep}{0in} \setlength{\partopsep}{0in}
      \setlength{\leftmargin}{0.17in}}}{\end{list}}
\newenvironment{list2}{
  \begin{list}{$\bullet$}{%
      \setlength{\itemsep}{0in}
      \setlength{\parsep}{0in} \setlength{\parskip}{0in}
      \setlength{\topsep}{0in} \setlength{\partopsep}{0in}
      \setlength{\leftmargin}{0.2in}}}{\end{list}}
    
\begin{document}

\name{\LARGE Andree Valle Campos}
\address{Calle Tambo Huascar 201, San Miguel, Lima-Peru}
\address{\textit{contact:} avallecam@gmail.com or (+51)950951722}

\begin{resume}

\section{\sc Interests}

Get a position in the field of Genomics, with an emphasis in high-throughput data analysis of transcriptomics and proteomics studies.\\

\section{\sc Education}
{\bf Universidad Nacional Mayor de San Marcos}, Lima-Peru \hfill Mar 2011 - Dec 2015\\
%\vspace*{-.1in}
BSc. Genetics and Biotechnology\\

\section{\sc Research Experience}
%%%%%%

{\bf U.S. Naval Medical Research Unit Six (NAMRU-6)}, Peru.\\
Department of Parasitology - Immunology and Vaccine Development Unit\\
\vspace*{-.1in}
\begin{list1}
	\item[] {\em Thesis Student} \hfill {\bf Ago 2015 - Present}\\
	\vspace*{-.1in}
	\begin{list2} %Job Description%
		\item Advisor: G. Christian Baldeviano, PhD
		\item Thesis Project: High-throughput Immunomics and Bioinformatics approach for the discovery of new antigenic determinants associated with protection against severe malaria\\
	\end{list2}
%%%%%%
	\vspace*{-.1in}
	\item[] {\em Trainee} \hfill {\bf Jan 2014 - Jul 2015}\\
	\vspace*{-.1in}
	\begin{list2} %Job Description%
		\item Experience: Bioinformatics (Phylogenetics, Structural Genomics), Biostatistics (Protein Microarray dataset analysis, drug-response assays), Molecular Biology (Gene cloning and protein expression), Immunoassays (DELI, ELISA)
	\end{list2}
\end{list1}

{\bf Universidad Nacional Mayor de San Marcos}, Lima, Peru.\\
Laboratory of Physiology and Animal Reproduction\\
\vspace*{-.1in}
\begin{list1}
	\item[] {\em Undergraduate researcher} \hfill {\bf Mar 2012 - Dec 2014}\\
	\vspace*{-.1in}
	\begin{list2} %Job Description%
		\item Advisor: Mg. Martha Valdivia Cuya
		\item 2014 Project: In vitro effect of ELF-Magnetic Fields on sperm motility of Alpaca\\
	\end{list2}
\end{list1}

\section{\sc Dry-Lab Skills}
{\bf Statistical packages}: Experienced in R and Bioconductor packages.\\
Applications: Microarray data analysis and visualization. Non-linear regression for parameter estimation in Dose-response curves. Exploratory analysis of Geographical Information Systems (GIS).\\
{\bf Programming Languages}: Familiar with Python.\\
Applications: Numerical integration for solving ODE and PDE. Population dynamics simulations.\\
{\bf Operating Systems}: Linux and Windows.\\
{\bf Bioinformatics softwares}: Phylogenetics (MrBayes, Arlequin), Structural Genomics (PyMol), Genome assembly (Artemis).

\section{\sc Wet-Lab Skills}
{\bf Molecular Biology}: Experienced in building of genetic constructs for gene cloning and protein expression. Familiar with analysis of gene expression by conventional and quantitative PCR.\\
{\bf Biochemistry}: Familiar with immunoassays, drug-response assays, stem-cell isolation, characterization, cryopreservation, and spermiogram test.\\

\section{\sc Personal Achievements}
{\bf Ranked 1st at the UNMSM Public University Admission Test} \hfill March 2011\\
Highest score among the Basic Sciences Faculties from a total of 1000 applicants.\vspace*{.05in}\\
{\bf Ranked 1st at the X Course and Workshop on Molecular Biology \\Techniques Applied to Infectious and Tropical Diseases} \hfill January 2013\\
Highest grade among 40 graduate and undergraduate students on the intensive summer course organized by the Universidad Peruana Cayetano Heredia (UPCH) Lima-Peru.\\

%%%%%%%%%%%%%%%%
\newpage
\section{\sc Poster presentation}

Quispe J., \textbf{Valle-Campos A.}, Ulloa G., Rodriguez L., Cruz V. and Valdivia M. "In vitro effect of Extremely Low Frequency Magnetic Field on the sperm motility of Alpacas: A preliminary study", {\em Annual Meeting of the Bioelectromagnetics Society - BioEM2015. Monterey, USA}, July 2015

Punil R., Murillo A., Carrasco M., Huaman A., Quispe J., Miranda J., Valladares K. and \textbf{Valle A}. "Dermatoglyphic analysis on individuals with Down syndrome and Autism in comparison to a control group", {\em XV National Congress of Biology Students - CONEBIOL 2014. Lima, Peru}, October 2014

Valdivia M., Tataje L., Cisneros S., Carmen R., Guillen W. de los Santos, Davila D., \textbf{Valle A.}, et.al. "Important genes for the camelid reproduction", {\em International Meeting of Research Groups in Basic and Applied Sciences - ASCILA 2012. Lima, Peru}, May 2012\\

\section{\sc International training}
{\bf School on Physics Applications in Biology}\hfill {January 2016}\\
Selected with complete financial support
\begin{list2} %Job Description%
	\item \textit{Experience:} Three-week school on the applications of Game Theory, Non linear Dynamics and Statistical Physics to biological problems.
	\item \textit{Organization:} International Center for Theoretical Physics - South American Institute for Fundamental Research (ICTP-SAIFR). IFT-UNESP, Sao Paulo- Brazil.
\end{list2}

{\bf V Southern-Summer School on Mathematical Biology}\hfill {January 2016}\\
Selected with complete financial support
\begin{list2} %Job Description%
	\item \textit{Experience:} One-week school on Population dynamics modeling applying differential calculus with a multidisciplinary approach (physics, mathematics, computational sciences and biology).
	\item \textit{Organization:} ICTP-SAIFR. IFT-UNESP, Sao Paulo- Brazil.
\end{list2}

{\bf Minischool on Biophysics of Protein Interactions \\and Onuchic Minicourse on Energy Landscapes}\hfill {March 2015}\\
Selected with complete financial support
\begin{list2} %Job Description%
	\item \textit{Experience:} One-week school on the theoretical, computational and experimental approaches to the study of protein folding, structure, kinetics and electrostatic effects of biomolecules.
	\item \textit{Organization:} ICTP-SAIFR. IFT-UNESP, Sao Paulo- Brazil.
\end{list2}

{\bf Latin-American training workshop on molecular epidemiology \\applied to infectious diseases}\hfill {November 2013}\\
Invited by the institution
\begin{list2} %Job Description%
	\item \textit{Experience:} Informatics tools for the epidemiological study. Last advances on molecular biology and epidemiology informatics. Sequencing and genotyping of leishmaniasis, tuberculosis and malaria infectious agents.
	\item \textit{Organization:} Institute of Tropical Medicine Antwerp (Belgium) and Instituto de Medicina Tropical “Alexander von Humboldt” (Peru). Lima - Peru.\\
\end{list2}

\section{\sc Teaching experience}
{\bf Biomathematics}:  {\em Discrete Modeling methods}. Applications to Gene Regulatory Network dynamics and Population dynamics. December 2015\\

\section{\sc References }

\begin{tabular}{ l l }
	G. Christian Baldeviano, PhD & Head, Immunology and Vaccine Development Unit \\
	Advisor & Naval Medical Research Unit Six (NAMRU-6)\\
	Jan 2014 - present & geralc.baldeviano.fn@mail.mil\\
	&\\
	Prof. Walter Cabrera-Febola & Chief, Group of Natural Structures and Theoretical Research \\
	Teacher and mentor & Universidad Nacional Mayor de San Marcos\\
	Mar 2013 - present & febcawal@gmail.com\\
	&\\
	Andres G. Lescano, MSH, PhD & Associate Professor and Masters' Program Coordinator\\
	Advisor & Universidad Peruana Cayetano Heredia\\
	Jan 2014 - Jun 2015 & andres.lescano.g@upch.pe\\
\end{tabular}

\newpage

\section{\scshape APPENDIX}

\section{\sc Complementary education}

{\bf \textit{In silico} Workshop: Techniques on molecular modeling of proteins}\hfill {June 2015}\\
\vspace*{-.1in}
\begin{list2} %Job Description%
	\item \textit{Organization:} Instituto Peruano de Genetica
	\item \textit{Time:} 8h
\end{list2}

{\bf Phylogenetics and Bioinformatics sequence analysis training - Level 1}\hfill {January 2015}\\
\vspace*{-.1in}
\begin{list2} %Job Description%
	\item \textit{Organization:} U.S. Naval Medical Research Unit Six (NAMRU-6).
	\item \textit{Time:} 30h
\end{list2}

{\bf Theory and software course: Gene Cloning}\hfill {December 2014}\\
\vspace*{-.1in}
\begin{list2} %Job Description%
	\item \textit{Organization:} Universidad Agraria La Molina.
	\item \textit{Time:} 12h
\end{list2}

{\bf Course on genomic analysis of microorganism, sequencing, \\assemble and annotation}\hfill {October 2014}\\
\vspace*{-.1in}
\begin{list2} %Job Description%
	\item \textit{Organization:} Universidad Nacional Mayor de San Marcos.
	\item \textit{Time:} 20h
\end{list2}

{\bf Basic Course-Workshop: Gene cloning and protein expression by \\recombinant DNA techniques}\hfill {November 2013}\\
\vspace*{-.1in}
\begin{list2} %Job Description%
	\item \textit{Organization:} Universidad Peruana Cayetano Heredia.
	\item \textit{Time:} 60h
\end{list2}

{\bf X Course and Workshop on Molecular Biology Techniques Applied \\to Infectious and Tropical Diseases}\hfill {January 2013}\\
\vspace*{-.1in}
\begin{list2} %Job Description%
	\item \textit{Organization:} Universidad Peruana Cayetano Heredia.
	\item \textit{Time:} 70h
\end{list2}

{\bf VI International campus course: Perspectives of reproductive \\technologies in the Andean Region}\hfill {January 2013}\\
\vspace*{-.1in}
\begin{list2} %Job Description%
	\item \textit{Organization:} Universidad Nacional Mayor de San Marcos.
	\item \textit{Time:} 30h\\
\end{list2}


\end{resume}
\end{document}




