\documentclass[margin,line]{res}

%%%%%%%%%%%%%%%%%%%%%%%%%%%%%%%%%%%%%%%%%%%%
%% ADD-IN to activate spanish text (tildes)
\usepackage[spanish]{babel}
\selectlanguage{spanish}
\usepackage[utf8]{inputenc}
%%%%%%%%%%%%%%%%%%%%%%%%%%%%%%%%%%%%%%%%%%%%

\oddsidemargin -.5in
\evensidemargin -.5in
\textwidth=6.0in
\itemsep=0in
\parsep=0in
\topmargin=0in
\topskip=0in
\textheight 10in

\newenvironment{list1}{
	\begin{list}{\ding{113}}{%
			\setlength{\itemsep}{0in}
			\setlength{\parsep}{0in} \setlength{\parskip}{0in}
			\setlength{\topsep}{0in} \setlength{\partopsep}{0in}
			\setlength{\leftmargin}{0.17in}}}{\end{list}}
\newenvironment{list2}{
	\begin{list}{$\bullet$}{%
			\setlength{\itemsep}{0in}
			\setlength{\parsep}{0in} \setlength{\parskip}{0in}
			\setlength{\topsep}{0in} \setlength{\partopsep}{0in}
			\setlength{\leftmargin}{0.2in}}}{\end{list}}

\begin{document}
	
	\name{\LARGE Andree Valle Campos}
	\address{Calle Tambo Huascar 201, San Miguel, Lima-Perú}
	\address{\textit{contacto:} avallecam@gmail.com o (+51)950951722}
	
	\begin{resume}
		
		\vspace*{.15in}
		
		\section{\sc Interés \\Científico}
		
		%Obtener una posición en el campo de la Ciencia de Datos y Análisis Estadístico de alta dimensión.\\
		%Aplicaciones a Genómica y Epidemiología siguiendo principios de Ciencia Reproducible.\\ 
		
		Biología cuantitativa e interdisciplinaria. Análisis Estadístico de datos siguiendo principios de\\ Ciencia Reproducible y Redes Biológicas para el modelamiento de sistemas dinámicos.\\%
		
		\section{\sc Educación}
		{\bf Universidad Nacional Mayor de San Marcos}, Lima-Perú \hfill Mar 2011 - Dic 2015\\
		Bachiller en Genética y Biotecnología\hfill Jun 2016\\
		Promedio: 14.09. GPA equivalente\footnotemark\textsuperscript{,}\footnotemark: 3.05\\%2.79
		\footnotetext[1]{https://www.wes.org/gradeconversionguide/index.asp}
		\footnotetext[2]{http://bioegrad.berkeley.edu/prospectivegrads/gpaconversion}
		
		\begin{center}
			\vspace{-9mm}
			\begin{tabular}{lll}
				\textit{Semestre} & \textit{Cursos relevantes} & \textit{Nota} [0-20] \\
				%		2012-I & General Biochemistry & 16\\
				%2012-II & General Physics II & 16\\
				%2013-I & Molecular Genetics & 16\\
				2013-II & Bioinformática & 16\\
				2013-II & Ecología Teórica & 16\\
				%		2014-I & General Systematics & 16\\
				2014-II & Física de Macromoléculas & 17\\
				2014-II & Biomatemática & 18\\
				%2015-II & Animal Biotechnology & 18\\
			\end{tabular}
			\vspace{4mm}
		\end{center}
		
		
		\section{\sc Experiencia en investigación}
		
		{\bf U.S. Naval Medical Research Unit Six (NAMRU-6)}, Perú.\\
		Departamento de Parasitología - Unidad de Inmunología y Desarrollo de Vacunas\\
		\vspace*{-.1in}
		\begin{list1}
			\item[] {\em Consultor} \hfill {\bf Nov 2016 - Feb 2017}\\
			\vspace*{-.1in}
			\begin{list2}
%				\item Inv.Principal.: Viviana Pinedo-Cancino, PhD. (UNAP)
				\item Trabajo: Estandarización de placas de ELISA para ensayos de sero-vigilancia usando R.\\
			\end{list2}
			%%%%%%
			\vspace*{-.1in}
			\item[] {\em Tesista} \hfill {\bf Ago 2015 - Dic 2016}\\
			\vspace*{-.1in}
			\begin{list2}
%				\item Asesor: G. Christian Baldeviano, PhD.
				\item Project: Perfil en larga escala de anticuerpos en respuesta a la infección de malaria vivax.\\
			\end{list2}
			%%%%%%
			\vspace*{-.1in}
			\item[] {\em Practicante} \hfill {\bf Ene 2014 - Jul 2015}\\
			\vspace*{-.1in}
			\begin{list2} %Job Description%
				\item Experiencia: Sesiones de revisión crítica de literatura e invest. computacional independiente.
			\end{list2}
		\end{list1}
		
		{\bf Universidad Nacional Mayor de San Marcos (UNMSM)}, Perú.\\
		Laboratorio de Fisiología y Reproducción Animal\\
		\vspace*{-.1in}
		\begin{list1}
			\item[] {\em Investigador de pregrado} \hfill {\bf Mar 2012 - Dic 2015}\\
			\vspace*{-.1in}
			\begin{list2}
%				\item Asesora: Mg. Martha Valdivia Cuya
				\item Experiencia: Escritura de propuestas de financiamiento y trabajo colaborativo.
				\item Proyecto: Efecto \textit{in vitro} de campos magnéticos ELF en la motilidad espermática de Alpaca.\\
			\end{list2}
		\end{list1}
		
		\section{\sc Habilidades en Ciencia de Datos}%Experienced in the ...
		{\bf Computación estadística}: R\\
		{\bf Bioestadística}: Paquete \texttt{limma} para la expresión diferencial en data de microarreglos.\\ 
		%Bioconductor
		{\bf Manipulación}: Paquetes del \texttt{tidyverse} para la preparación de bases de datos epidemiológicas.\\
		{\bf Visualización}: Gráficos en \texttt{base} y \texttt{ggplot2}. Diagramas de flujo con \texttt{DiagrammeR}.\\ 	
		
		\section{\sc Habilidades computacionales}
		{\bf Lenguaje de programación}: Python\\%, Bash (Unix shell).\\%Perl, 
		{\bf Bioinformática}: biología molecular \textit{in silico}, filogenética y genómica estructural.\\	%ensamble de genomas
		{\bf Preparación de documentos}: LaTeX, R Markdown.\\
		%usando el paquete \texttt{knitr} y Pandoc
		{\bf Sistema Operativo}: GNU/Linux (distribución Ubuntu).\\ %%, Windows, OSX		
		{\bf Control de versiones}: Git.\\ %% ADD LINK!! %%
%		{\bf Edición y procesamiento de Imágenes}: ImageJ, GIMP, Photoshop.\\
		
		\section{\sc Habilidades Experimentales}
		{\bf Biología Molecular}: Clonamiento, expresión de proteínas, PCR convencional y quantitativo.\\
		{\bf Bioquímica}: Criopreservación de células madre y ensayos basados en ELISA.\\
		
		%%%%%%%%%%%%%%%%%%%%%%%%%%%%%%%%%%%%%%%%%%%%%%%%%
		\newpage
		
		\section{\sc Logros personales}
		
		{\bf 1\textsuperscript{er} puesto en el X Curso y Workshop sobre Biología Molecular \\Técnicas aplicadas a enfermedades infecciosas y tropicales. UPCH} \hfill Enero 2013\\
		Mayor calificación entre 40 estudiantes de posgrado y pregrado.
		
		{\bf 1\textsuperscript{er} puesto en el exámen general de admisión. UNMSM} \hfill Marzo 2011\\
		Mayor calificación entre 1000 aplicantes para las Facultades de Ciencias Básicas.\\
		
		%\section{\sc International training}
		\section{\sc Educación complementaria}
		
		{\bf Wellcome Genome Campus Advanced Courses:\\Working with Parasite Data Resources}\hfill {Octubre 2016}\\
		\vspace*{-.1in}%Complete financial support
		\begin{list2} %Job Description%
			\item Aplicaciones de \texttt{eupathdb.org} en genómica, proteómica y metabolómica.
			\item Wellcome Trust Sanger Institute, UK. Montevideo - Uruguay.
%			\item \textit{Lugar:} Instituto del Higiene, Montevideo, Uruguay.%\\
%			\item \textit{Organizado por:} EuPathDB y Wellcome Trust Sanger Institute, UK. 
			%\\ Resource Center
			%the Eukaryotic Pathogen Bioinformatics Resource Center 
			%	\item \textit{Time:} 35h
		\end{list2}
		
		{\bf Workshop EPONGE: Epidemiology meets POpulation GEnetics}\hfill {Octubre 2016}\\
		\vspace*{-.1in}
		\begin{list2} %Job Description%
			\item Tópicos introductorios y actualización. Aplicación de inferencia bayesiana.
			\item University of Antwerp y UPCH. Lima - Perú.
%			\item \textit{Organizado por:} Global Health Institute - University of Antwerp y UPCH 
			%and UNAP.%\\
			%\item \textit{Time:} 30h
		\end{list2}

		%{\bf School on Physics Applications in Biology}\hfill {January 2016}\\
		%Selected with complete financial support
		%\begin{list2} %Job Description%
		%	\item \textit{Experience:} Three-week school on Game theory, Non-linear dynamics and Statistical physics.
		%	\item \textit{Organized by:} ICTP-SAIFR. IFT-UNESP, Sao Paulo - Brazil. %International Center for Theoretical Physics - South American Institute for Fundamental Research (ICTP-SAIFR). IFT-UNESP, Sao Paulo- Brazil.
		%\end{list2}
		
		{\bf V Southern-Summer School on Mathematical Biology}\hfill {Enero 2016}\\
		\vspace*{-.1in}%Complete financial support
		\begin{list2} %Job Description%
			\item Modelamiento en dinámica de poblaciones en Ecología y Epidemiología.
			\item ICTP-SAIFR. IFT-UNESP, Sao Paulo - Brazil.
%			\item \textit{Experiencia:} Escuela sobre el modelamiento de dinámica de poblaciones.
%			\item \textit{Organizado por:} ICTP-SAIFR. IFT-UNESP, Sao Paulo - Brazil.
		\end{list2}
		
		{\bf Phylogenetics and Bioinformatics sequence analysis training - Level 1}\hfill {Enero 2015}\\
		\vspace*{-.1in}
		\begin{list2} %Job Description%
			\item Curso introductorio con aplicaciones en Dengue, VIH, tuberculosis y malaria.
			\item U.S. Naval Medical Research Unit Six (NAMRU-6).%\\
%			\item \textit{Organizado por:} U.S. Naval Medical Research Unit Six (NAMRU-6).%\\
			%\item \textit{Time:} 30h
		\end{list2}
		
		%{\bf Minischool on Biophysics of Protein Interactions \\and Onuchic Minicourse on Energy Landscapes}\hfill {March 2015}\\
		%Selected with complete financial support
		%\begin{list2} %Job Description%
		%	\item \textit{Experience:} One-week school on protein folding and electrostatic effects in biomolecules.
		%	\item \textit{Organized by:} ICTP-SAIFR. IFT-UNESP, Sao Paulo - Brazil.
		%\end{list2}
		
		%{\bf Theory and software course: Gene Cloning}\hfill {December 2014}%\\
		%\begin{list2} %Job Description%
		%	\item \textit{Venue:} Universidad Agraria La Molina.%\\
		%\textit{Time:} 12h
		%\end{list2}
		
		
		%{\bf Course on genomic analysis of microorganism, sequencing, \\assemble and annotation}\hfill {October 2014}%\\
		%\begin{list2} %Job Description%
		%	\item \textit{Venue:} Universidad Nacional Mayor de San Marcos.%\\
		%\textit{Time:} 20h
		%\end{list2}
		
		{\bf Latin-American training workshop on molecular epidemiology \\applied to infectious diseases}\hfill {Noviembre 2013}\\
		\vspace*{-.1in}%Invited by the institution
		\begin{list2} %Job Description%
			\item Genotipificación e Informática aplicada a tuberculosis, leishmania y malaria.
			\item ITM Antwerp y IMTAvH-UPCH, Lima - Perú.\\
%			\item \textit{Experiencia:} Genotipificación e Informática aplicada a tuberculosis y malaria. 
			%leishmania, 
%			\item \textit{Organizado por:} ITM Antwerp (Belgica) y IMTAvH (Perú), Lima - Perú.\\
		\end{list2}
		
		
		
		\section{\sc Presentación de cartel}
		
		Quispe J., \textbf{Valle-Campos A.}, Ulloa G., Rodriguez L., Granados E., Cruz V., Valdivia M, et al.\\ ``In vitro effect of Extremely Low Frequency Magnetic Field on the sperm motility of Alpacas", \\ {\em Annual Meeting of the Bioelectromagnetics Society - BioEM2015. Monterey, USA}, July 2015\\
		%Liñan A., Limaymanta O., Fuentes P., Carhuaricra D., %: A preliminary study
		
		%Punil R., Murillo A., Carrasco M., Huaman A., Quispe J., Miranda J., Valladares K. and \textbf{Valle A}. ``Dermatoglyphic analysis on individuals with Down syndrome and Autism in comparison to a control group", {\em XV National Congress of Biology Students - CONEBIOL 2014. Lima, Peru}, October 2014
		
		%Valdivia M., Tataje L., Cisneros S., Carmen R., Guillen W. de los Santos, Davila D., \textbf{Valle A.}, et.al. ``Important genes for the camelid reproduction", {\em International Meeting of Research Groups in Basic and Applied Sciences - ASCILA 2012. Lima, Peru}, May 2012\\
		
		
		\section{\sc Experiencia en enseñanza}
		%Invited teacher for the following third-year undergraduate elective courses: \\
		%{\bf Biomathematics}:  {\em Population dynamics}. \hfill {December 2015}\\ Discrete modeling methods in Ecology. \\ %%Graph theory and Linear algebra 
		{\bf Biomatemática}: {\em Redes de Regulación Génica}. \hfill {Diciembre 2015/16}\\ Topología y dinámica de redes empleando Teoría de Grafos y autómatas finitos. \\[4pt] 
		%% Graph theory to model Topology and Automata Networks for Dynamics of GRN %% Modeling of Networks
		{\bf Transferencia Genética Horizontal}: {\em Sobre el \#tardigate y bioinformática}. \hfill {Diciembre 2016}\\ Revisión de la controversia sobre el primer genoma de tardígrados.\\
		
		
		\section{\sc Idiomas}
		{\bf Inglés}: {\em Nivel Avanzado completo}. Instituto BRITÁNICO. \hfill {Set 2012 - Feb 2014} \\
		
		
		\section{\sc Referencias}
		
		\begin{tabular}{ l l }
			G. Christian Baldeviano, PhD & Jefe, Unidad de Inmunología y Desarrollo de Vacunas\\
			Asesor & Naval Medical Research Unit Six (NAMRU-6)\\
			Ene 2014 - presente & geralc.baldeviano.fn@mail.mil\\
			&\\
			Andres G. Lescano, MSH, PhD & Profesor Asociado y Coordinador del Programa de Maestría\\
			Supervisor & Universidad Peruana Cayetano Heredia (UPCH)\\
			Ene 2014 - Jun 2015 & andres.lescano.g@upch.pe\\
			&\\
			Prof. Walter Cabrera-Febola & Jefe, Grupo en Estructural Naturales e Investigación Teórica\\
			Profesor & Universidad Nacional Mayor de San Marcos (UNMSM)\\
			Mar 2013 - presente & febcawal@gmail.com\\	
		\end{tabular}
		
		%%%%%%%%%%%%%%%%%%%%%%%%%%%%%%%%%%%%%%%
		%\newpage
		
		%\section{\scshape APPENDIX }
		
		%{\scshape \textbf{COMPLEMENTARY EDUCATION}}
		
		%\section{\sc Computational courses}
		
		
		
		%{\bf \textit{In silico} Workshop: Techniques on molecular modeling of proteins}\hfill {June 2015}\\
		%\textit{Organized by:} Instituto Peruano de Genetica - IPEGEN\\
		%\textit{Time:} 8h
		
		
		
		
		
		
		
		
		%\section{\sc Experimental courses}
		
		%{\bf X Course and Workshop on Molecular Biology Techniques Applied \\to Infectious and Tropical Diseases}\hfill {January 2013}\\
		%Achievement: Ranked 1\textsuperscript{st}.\\
		%\textit{Organized by:} IMTAvH, Unidad de Epidemiologia Molecular - UPCH.\\
		%\textit{Venue:} Universidad Peruana Cayetano Heredia.\\
		%\textit{Time:} 70h
		
		%{\bf VI International campus course: Perspectives of reproductive \\technologies in the Andean Region}\hfill {January 2013}\\
		%\textit{Organized by:} Vrije Universiteit Brussel (Belgium) and UNMSM.\\
		%\textit{Time:} 30h\\
		
		%{\bf Basic Course-Workshop: Gene cloning and protein expression by \\recombinant DNA techniques}\hfill {August 2012}\\
		%\textit{Organized by:} Lab. Bioinformatica y Biologia Molecular - UPCH.\\
		%\textit{Venue:} Universidad Peruana Cayetano Heredia.\\
		%\textit{Time:} 60h
		
		
		
		
	\end{resume}
\end{document}




